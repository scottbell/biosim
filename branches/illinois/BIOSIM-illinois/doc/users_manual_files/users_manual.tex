\documentstyle{report}

\input epsf

%\nofiles

\setlength{\textheight}{9.0in}
\setlength{\textwidth}{6.5in}
\setlength{\footheight}{0.0in}
\setlength{\topmargin}{0.00in}
\setlength{\headheight}{0.0in}
\setlength{\headsep}{0.0in}
\setlength{\oddsidemargin}{-0.0in}

\makeindex

\begin{document}

\def\topfraction{1.0}        %maximum fraction of floats at the top of the page
\def\bottomfraction{1.0}     %ditto, for the bottom of the page
\def\textfraction{0}         %minimum fraction of text (--> 100% floats is okay)
\def\floatpagefraction{1.0}  % if a page is full of floats, it'd better be FULL

\newcommand{\insertfigure}[2]{\begin{center}
                             \ \epsfxsize=#2 \epsfbox{#1.ps}
                             \end{center}}

%\newcommand{\insertfigure}[2]{\vspace{#2}}

\title{BioSim: An Integrated Simulation of an Advanced Life Support System for Intelligent Control Research\\
Users Manual (draft)}

\author{Metrica Inc. and S\&K Technologies\\
NASA Johnson Space Center/ER2\\
Houston TX 77058\\
http://www.traclabs.com/biosim}


\date{\today}

\maketitle

\tableofcontents

\newpage

\listoffigures

\newpage

\chapter{Introduction}
Advanced life support systems have multiple interacting subsystems,
which makes control a particularly challenging task.  The simulation
described in this document provides a testbed for integrated control
research.  There have been other integrated life support simulations
(e.g., \cite{finn99}) and we have learned from those efforts.  This
simulation is designed exclusively for integrated controls research,
which imposes different requirements.  For example, the simulation is
accessed through sensors and actuators, just as a real system would
be.  Noise and uncertainty are built in and controllable.
Malfunctions and failures of subsystems are modeled and manifest
themselves through anomalous readings in the sensors.  Crew members
are taskable and their tasks have purpose and meaning in the
simulation.  In essence, the simulation is a replacement for the
Advanced Life Support (ALS) hardware and crew, allowing for testing of
control approaches in advance of any integrated test.  We want the
simulation to be used to develop and evaluate integrated control
techniques.  There are still many open research questions with respect
to controlling advanced life support systems.  For example is a
distributed or hierarchical approach better?  What role does machine
learning play in control?  How can symbolic, qualitative control
approaches be integrated with continuous, quantitative approaches?
How can we evaluate different control philosophies?  We hope that the
controls community can use this simulation to begin to build systems
which will provide us with answers to these kinds of questions. 

The simulation is written in Java and is accessed using the Common
Object Request Broker Architecture (CORBA)\index{CORBA}.  The
distributed nature of the simulation allows for multiple instances to
be run in parallel, making it ideal for testing advanced control
concepts such as genetic algorithms or reinforcement learning, which
require multiple trials.  See \cite{kortenkamp_etal01} or
\cite{schreckenghost_etal02} for a background on advanced life support
control issues and previous work.  This document begins with a brief
introduction to advanced life support systems and provides details on
installing, running and interacting with the simulation.  The
simulation is available from:

\begin{verbatim}
     http://www.traclabs.com/biosim
\end{verbatim}

\chapter{Background on advanced life support}

An advanced life support system is one in which many resources are
reused or regenerated.  Such systems will be necessary for long
duration human missions, such as those to Mars, where resupply
opportunities are limited.  Typically they have thin margins for mass,
power and buffers, which requires optimization and tight control.
Also, advanced life support systems consist of many interconnected
subsystems all of which interact in both predictable and unpredictable
ways (see Figure~\ref{fig:biosim}).  Autonomous control\index{control}
of these systems is desirable. This section gives enough background so
that users of the simulation can understand the major components of an
advanced life support system.  For detailed documentation on advanced
life support systems see \cite{tri99} or go to:

\begin{verbatim}
    http://advlifesupport.jsc.nasa.gov
\end{verbatim}

\begin{figure}
\insertfigure{figure}{5.5in}
\caption{The various modules that comprise the BIOSim integrated simulation.}
\label{fig:biosim}
\end{figure}

\section{Modules}

An advanced life support system consists of many interacting
subsystems.  While each is self-contained, they rely on each other for
various resources.  The simulation is built from a set of modules that
each produce and consume different resources.

\subsection{Environment}
\label{sec:env}

An environment\index{environment} contains air that is consumed by
either people or crops.  Air\index{air} contains a mixture of gases --
in our simulation these gases are oxygen (O2), carbon dioxide (CO2),
nitrogen (N), water vapor (H20), and other gasses (trace).  The
initial composition of the gases is set by the simulation
initialization file (see Section~\ref{sec:init}).  As the simulation
runs modules may consume air from the simulation and replace it with
air of a different composition.  Thus, the composition of gases and
pressure in the air change over time and can be measured by
environment sensors (Section~\ref{sec:env_sen}).  As with all modules,
there can be multiple environments.  For example, it is common for
crew members and crops to have different air compositions and that is
the default in this simulation.

\subsection{Crew}
\label{sec:crew}

The crew\index{crew} module is implemented using models described in
\cite{goudarzi_ting99}.  The number, gender, age and weight of the
crew are settable as input parameters -- in the default configuration
there are four crew members, two male and two female.  The crew cycles
through a set of activities (sleep, maintenance, recreation, etc.).
As they do so they consume O2, food and water and produce CO2, dirty
water and solid waste.  The amount of resources consumed and produced
varies according to crew member attributes and their activities.  The
crew's activities can be adjusted by passing a new crew schedule to
the crew module.  A default schedule can also be used.  The crew
module is connected to a crew environment that contains an atmosphere
that they breathe.  The initial size and gas composition (percentages
of O2, CO2, H2O and inert gases) are input parameters and the default
is $1.54893 x 10^6 liters$ (from \cite{tri99}) with an atmosphere
equivalent to sea level air.  As the simulation progresses the mixture
of gases in the atmosphere changes.

\subsection{Water}
\label{sec:water}
Consumes:Power, Grey Water, Dirty Water
Produces: Potable Water

The water\index{water} recovery module consumes dirty water, grey
water(i.e., water that can be used for washing but not drinking), and
power and produces potable water.  The water recovery module consists
of four subsystems that process the water.  The biological water
processing (BWP) subsystem removes organic compounds.  Then the water
passes to a reverse osmosis (RO) subsystem, which makes 85\% of the
water passing through it grey.  The 15\% of water remaining from the
RO (called brine) is passed to the air evaporation subsystem (AES),
which recovers the rest.  These two streams of grey water (from the RO
and the AES) are passed through a post-processing subsystem (PPS) to
create potable water.  An external controller can turn on or off
various subsystems.  For example, all water can pass through the AES
at a higher power cost.  We based our water recovery module on a
recently completed test at NASA Johnson Space Center
\cite{bonasso_etal03}.

\subsection{Air}
\label{sec:air}
Consumes:Power, H2, Air, CO2, Potable Water
Produces: Air (with less CO2), O2, H2, CO2, Potable Water

The air\index{air} component takes in exhalant CO2 and produces O2 as long as
there is sufficient power being provided to the system.  This
component is modeled on various Air Revitalization System (ARS) work
at NASA JSC \cite{malin_etal00}.  There are three interacting air
subsystems: the Variable Configuration Carbon Dioxide Removal (VCCR)
System in which CO2 is removed from the air stream; the Carbon Dioxide
Reduction System (CRS), which also removes CO2 from the air stream
using a different process and producing different gases than the VCCR;
and the Oxygen Generation System (OGS) in which O2 is added to the air
stream by breaking water down into hydrogen and oxygen.  It is
important to note that both the removal of CO2 and the addition of O2
are required for human survival.  It is also important to note that
the biomass component (next subsection) also removes CO2 and adds O2.

\subsection{Biomass}
\label{sec:biomass}
Consumes:Power, Potable Water, Grey Water, Air
Produces: Air (with more CO2), Biomass, Dirty Water, CO2, Potable Water

The biomass\index{biomass} component is where crops\index{crops} are
grown.  It produces both biomass, which can be turned into
food\index{food}, and O2 and consumes water, power (light) and CO2.
This component is based on models given in \cite{jones_cavazzoni00}.
The system is modeled as shelves\index{shelves} that contain plants,
lights and water.  Shelves are planted and harvested and there is
growth cycle for each shelf.  Currently, ten crops are modeled and can
be planted in any ratio.

\subsection{Food processing}
\label{sec:food}
Consumes:Power, Biomass
Produces: Food

Before biomass can be consumed by the crew it must be converted to
food\index{food}.  The food processing component takes biomass, power and crew
time and produces food and solid waste.  The crew needs to be involved
in this process as it is labor intensive.  See Section~\ref{sec:crew}
for more information on scheduling crew activities.

\subsection{Waste}
\label{sec:waste}
Consumes:Power, Dry Waste, O2
Produces: CO2

The waste\index{waste} component consumes power, O2 and solid waste
and produces CO2.  It is modeled on an incinerator used in the Phase
III test in 1997 \cite{schreckenghost_etal02}.  Incineration can be
scheduled.

\subsection{Thermal control}
\label{sec:thermal}
Consumes: ?
Produces: ?

The thermal control\index{thermal} component regulates the air
temperature in the habitat.  This component is not implemented for
this simulation and it is assumed that external controllers are
maintaining chamber temperature.

\subsection{Power}
\label{sec:power}
Consumes: 
Produces: Power

The power\index{power} component supplies power to all of the other
components that need it.  There are two choices for power in the
simulation.  The first is nuclear\index{nuclear} power, which supplies
a fixed amount throughout the lifetime of the simulation.  The second
is solar\index{solar} power, which supplies a varying amount
(day/night cycle) of power to each component.

\subsection{Accumulators}
\label{sec:accumulator}
Consumes: All
Produces: All

The accumulator\index{accumulator} component can take a resource from any
store or environment and place it into another environment or store.  It is
functionally equivalent to an injector.

\subsection{Injectors}
\label{sec:injector}
Consumes: All
Produces: All

The injector\index{injector} component can take a resource from any
store or environment and place it into another environment or store.  It is
functionally equivalent to an accumulator.

\chapter{Simulation Properties}

The simulation implements the modules outlined in the previous
section.  BioSim does {\em not} simulate at the level of valves,
pumps, switches, etc.  Instead, modules are implemented in a
producer/consumer relationship, which is described in the next
subsection.  The simulation also provides additional functionality to
help test and debug control programs.

\section{Producer/Consumer Model}
\label{sec:pc}

No component in BioSim directly interacts with other components.
Instead, each component has a rigid set of resources that it consumes
and produces.  The resources are taken from stores/environments and
put into stores/environments.  The resources for BioSim currently
consist of power, potable water, grey water, dirty water, air, H2,
nitrogen, O2, CO2, biomass, food, and dry waste.  At each simulation
tick, each module takes resource from it's store (it's consumables),
and puts resources into stores (it's products).  Stores always report
thier values (level and capacity) from the last tick and report new
values after every component in the simulation has been ticked.
Resource conflicts can occur when two components ask for a limited
resource, i.e. two WaterRS's both need 20 liters of dirty water and
there is only 20 liters in the dirty water tank.  This is currently
resolved on a winner-take-all basis that is for all practical purposes
randomized.

\section{Success criteria}

In order to compare different control approaches there should be
objective criteria for a successful advanced life support system
mission.  Several possibilities exist already in the simulation.  For
example, the length of the mission before consumables are gone is a
success criterion.  As is minimizing the starting levels of stores or
minimizing the sizes of intermediary buffers.  However, many of these
fail to get at the true success of a mission, the productivity of the
crew in performing their science objectives.  Thus, the simulation
includes an artificial productivity\index{productivity} measure.  As
the crew cycles through their activities (see Section~\ref{sec:activities})
the amount of time they spend doing "mission" tasks is accumulated. This
number is also multiplied by a factor that takes into account the
amount of sleep, exercise, water, food and oxygen they are getting to approximate 
crew effectivness at performing the task (a happy crew is a productive crew!).
So each tick, for each crew member the productivity is given by the
following equation:

        [need to put the equation here]
	[some relevant code, derive equation later]
	\begin{verbatim}
	private static final float WATER_TILL_DEAD = 8.1f;
	private static final float WATER_RECOVERY_RATE=0.01f;
	private static final float OXYGEN_TILL_DEAD = 3f;
	private static final float OXYGEN_RECOVERY_RATE=0.01f;
	private static final float CALORIE_TILL_DEAD = 180000f;
	private static final float CALORIE_RECOVERY_RATE=0.0001f;
	private static final float DANGEROUS_CO2_RATION = 0.06f;
	private static final float CO2_TILL_DEAD = 10f;
	private static final float CO2_RECOVERY_RATE=0.001f;
	private static final float LEISURE_TILL_BURNOUT = 168f;
	private static final float LEISURE_RECOVERY_RATE=90f;
	private static final float AWAKE_TILL_EXHAUSTION = 120f;
	private static final float SLEEP_RECOVERY_RATE=12f;
	consumedWaterBuffer = new SimpleBuffer(WATER_TILL_DEAD, WATER_TILL_DEAD);
	consumedOxygenBuffer = new SimpleBuffer(OXYGEN_TILL_DEAD, OXYGEN_TILL_DEAD);
	consumedCaloriesBuffer = new SimpleBuffer(CALORIE_TILL_DEAD, CALORIE_TILL_DEAD);
	consumedCO2Buffer = new SimpleBuffer(CO2_TILL_DEAD, CO2_TILL_DEAD);
	sleepBuffer = new SimpleBuffer(AWAKE_TILL_EXHAUSTION, AWAKE_TILL_EXHAUSTION);
	leisureBuffer = new SimpleBuffer(LEISURE_TILL_BURNOUT, LEISURE_TILL_BURNOUT);
	float caloriePercentFull = sigmoidLikeProbability(consumedCaloriesBuffer.getLevel() / consumedCaloriesBuffer.getCapacity());
	float waterPercentFull = sigmoidLikeProbability(consumedWaterBuffer.getLevel() / consumedWaterBuffer.getCapacity());
	float oxygenPercentFull = sigmoidLikeProbability(consumedOxygenBuffer.getLevel() / consumedOxygenBuffer.getCapacity());
	float CO2PercentFull = sigmoidLikeProbability(consumedCO2Buffer.getLevel() / consumedCO2Buffer.getCapacity());
	float sleepPercentFull = sleepBuffer.getLevel() / sleepBuffer.getCapacity();
	float leisurePercentFull = leisureBuffer.getLevel() / leisureBuffer.getCapacity();
	float averagePercentFull = (caloriePercentFull + waterPercentFull + oxygenPercentFull + CO2PercentFull + sleepPercentFull + leisurePercentFull) / 6f;
	myMissionProductivity += myCrewGroup.randomFilter(averagePercentFull);
	\end{verbatim}
\section{Stochastic processes}

Any sufficiently complex process, especially one with biological
components, will not be deterministic\index{deterministic}.  That is,
given the same starting conditions and inputs it will not produce the
exact same outputs each time it is run.  A
stochastic\index{stochastic} process is one in which chance or
probability affect the outcome.  This simulation offers both
deterministic and stochastic operations.  In the former case, running
the simulation twice with the same inputs and initial conditions will
produce the same results.  This may be useful for quantitatively
comparing different control approaches against each other.  In the
latter case, the user can control the amount of variability in the
simulation (see Section 7.4).  This is modeled using a
Gaussian\index{Gaussian} distribution.  The deviation is determined by
the stochastic intensity; a higher intensity will yield a higher
deviation.  The filter is then appropriately applied to certain
variable of the model.  This can be used to test a control system�s
ability to deal with stochastic processes.

\section{Malfunctions}

The simulation also has the ability to accept
malfunction\index{malfunctions} requests (see Section 7.5).  These
requests will change the operating regime of the simulation.  For
example, by causing a crew member to be sick, power supplies to drop,
water to leak, plants to die, etc.  Different modules will have
different sensitivities, for example once plants die they cannot
recover, but if the water module is damaged and then repaired it will
operate normally.  Of course, if levels of CO2, water or food reach
hard-coded critical levels the crew will abandon the mission and return
home.

\section{Crew activity scheduling}
\label{sec:activities}
Crew members in an advanced life support system do not just consume
and produce -- they do activities.  These may be science, maintenance,
exercise, sleep, etc.  The different activities go for a specified
length and for a specified intensity.  The intensity is tied to how
many resources the crew member will consume performing the activity
(e.g., sleep takes less 02 to perform than exercise).  The following
shows the nominal daily routine for a crew member:

\begin{enumerate}
\item 	Sleep for 8 hours
\item	Hygiene for 1 hour
\item	Exercise for 1 hour
\item	Eating for 1 hour
\item	Mission tasks for 8 hours
\item	Health for 1 hour
\item	Maintenance for 1 hour
\item	Leisure for 2 hours
\end{enumerate}

This schedule can be interrupted by malfunctions (a crew member's
activity changed to repair), sickness, or even mission end (if the
crew member hasn't recieved proper resources).

\section{Mass balance}

Mass balanced means resources consumed by components are returned in
totality (but usually in a different form) to stores, no mass is lost.
This is important for closed loop simulations as it verfies the models
(under perfect conditions) are keeping with the conversation of energy
and mass.  Our simulation has been mass balanced to a reasonable
degree.  There is a slight loss in mass from the crew (they don't gain
weight from eating or produce heat) and from the plants (who don't
produce heat).  We plan to address these issues in the future, but the
simulation is currently mass balanced enough to provide a very decent
closed loop simulation.


\chapter{Simulation implementation}

All of the simulation components are written in Java\index{Java} to
make the simulation portable.  Each component can be run as it's own
CORBA\index{CORBA} server so that they can be distributed across
computers for increased performance, or the entire simulation can be
run on one server.  We have tested the simulation on
Windows\index{Windows}, Unix\index{Unix} and
Macintosh\index{Macintosh} computers.  A user interface allows for
insight into the simulation which provides real time graphing and
more.  There is also a logging\index{logging} facility that will
automatically save all simulation values to a file (xml) or a database
(in development).  We have written Java and C++\index{C++} interfaces
to the simulation, though any language that supports CORBA should be
able to access the simulation.

\begin{verbatim}
(BioSim) <-> Server ORB <-> Client ORB <-> (Client)
\end{verbatim}

I think we should provide a link to a CORBA tutorial, but assume
knowledge of basics of CORBA.  Explain object oriented methodology and
how it relates to BioSim.  Would a figure be good here showing the
relationship between servers, clients, CORBA, etc.?

\chapter{Installing the simulation}
The simulation is written in Java and can run on most popular
operating systems.  We provide both an executable for Windows and
source code for any operating system.  This chapter deals with
installing the code onto your computer. 

\section{Installing an executable}

An executable that will run under most Windows operating systems is
available.  You can download the Windows executable from: {\tt
  www.traclabs.com/biosim}.

At that site, go to Download and then download the Windows
installer\index{installation}.  Remember where you put it.
Double-click on the Setup.exe icon, accept the agreement, choose the
default installation, install it where you desire and then click
``Finish''.  BIOSim will start automatically.  NOTE: If you do not
have Java\index{Java} installed on your PC then the installer will
direct you to the WWW page to download it (free).  Please e-mail Scott
Bell (scott@traclabs.com) if you have trouble with installing the
simulation.

\begin{figure}
\insertfigure{labelled_interface}{5.5in}
\caption{BioSim�s graphical user interface.}
\label{fig:empty_interface}
\end{figure}

\section{Installing source code}

Please read the README.txt file in the doc directory of your BIOSIM distribution.

\chapter{Running the simulation}
\label{sec:running}

Doubleclicking on the ``BioSim'' icon in Windows will start the
simulation.  In Linux, run the {\tt run-biosim.sh} shell script in the
{\tt /bin} directory.  When BioSim starts you'll see a blank
interface\index{interface} as shown in
Figure~\ref{fig:empty_interface}.  Holding your mouse over the buttons
on the top reveals what they do.  As noted in the figure, the first
three buttons control the simulation.  The next series of buttons will
display various subsystem views. Some of the views, like the crew, do
not contain anything until the simulation is started.  It's often
easiest to open up the main subsystem views before starting the
simulation.

The open interface is shown in Figure~\ref{fig:full_interface}.  The
view can be changed between text, chart and schematic by clicking on
the tabs.  The simulation will run until resources (air, water, food,
etc.) drop below critical levels.  Then the simulation will stop.
Pressing the start button will restart the simulation from the
beginning.  If you want to examine the interfaces closely press the
``Pause'' button or go through the simulation one step at a time.

\begin{figure}
\insertfigure{open_interface}{5.5in}
\caption{A full view of BioSim�s graphical user interface}
\label{fig:full_interface}
\end{figure}

\chapter{Configuring the simulation}
The previous chapter discussed installing and running the default
BioSim.  This section will discuss configuring BioSim to address
specific user interests.  This includes changing the underlying
producer/consumer relationship, changing the stochastic nature of
BioSim and changing the logging of data.  We start with some basic
CORBA infrastructure requirements.     

\section{CORBA, Orbs and namesevers}

BioSim is a server.  Client programs connect to BioSim via
CORBA\index{CORBA} -- the Common Object Request Broker Architecture.
In order to connect to the BioSim server will you will need to install
an Object Request Broker (ORB)\index{ORB} on your machine.  We
recommend the JacORB for Java (http://www.jacorb.org) and the ACEOrb
for C++ (http://www.cs.wustl.edu/~schmidt/TAO.html), although any
CORBA-compliant ORB will work.  When the BioSim server is started it
registers with a nameserver\index{nameserver}, which tells client
programs where to find it.  The ORBs we recommend include a name
server.  This manual will not cover the details of CORBA, we refer you
to any book on CORBA for your language or look at the sample programs
in Appendix A and B.  We also assume that you have already installed
one of the ORBs.
   
\section{Initial conditions}
\label{sec:init}
BioSim is almost infinitely reconfigurable\index{recongifuration} --
from the number of crew members to the size and number of the
different modules and should be able to provide scenarios ranging from
transit vehicles to Mars colonies.  The start-up configuration of
BioSim is controlled by an eXtensible Markup Language (XML)\index{XML}
file that is read in during BioSim initialization.  If you want to
change BioSim's configuration you need to change the XML file.  The
default XML file is located in:

\begin{verbatim}
   BIOSIM/resources/biosim/server/framework/DefaultInit.xml
\end{verbatim}

\noindent where {\tt BIOSIM} is the directory where you installed
BioSim.

You can copy this file, change it's name, modify it and then give its
full path name as a parameter to BioSim on startup.  The flag when you
start BioSim is {\tt -xml} followed by the full path name of the XML
file.  If you do not wish to change the initial configuration of
BioSim you can skip the rest of this section. 

\subsection{XML file format}
A snippet from the default initialization XML file is shown below:

\begin{verbatim}
  <SimBioModules>
    <crew>
      <CrewGroup name="CrewGroup">
        <potableWaterConsumer maxFlowRates="100" desiredFlowRates="100" inputs="PotableWaterStore"/>
        <airConsumer maxFlowRates="0" desiredFlowRates="0" inputs="CrewEnvironment"/>
        <foodConsumer maxFlowRates="100'' desiredFlowRates="100" inputs="FoodStore"/>
        <dirtyWaterProducer maxFlowRates="100" desiredFlowRates="100" outputs="DirtyWaterStore"/>
        <greyWaterProducer maxFlowRates="100" desiredFlowRates="100" outputs="GreyWaterStore"/>
        <airProducer maxFlowRates="0" desiredFlowRates="0" outputs="CrewEnvironment"/>
         .
         .
         .
        <crewPerson name="Bob Roberts" age="43" weight="77" sex="MALE">
          <schedule>
            <activity name="sleep" length="8" intensity="1"/>
            <activity name="hygiene" length="1" intensity="2"/>
            <activity name="exercise" length="1" intensity="5"/>
            <activity name="eating" length="1" intensity="2"/>
            <activity name="mission" length="9" intensity="3"/>
            <activity name="health" length="1" intensity="2"/>
            <activity name="maintenance" length="1" intensity="2"/>
            <activity name="leisure" length="2" intensity="2"/>
          </schedule>
        </crewPerson>                               
         .
         .
         .
      </crew>
\end{verbatim}

This file implements the producer/consumer relationship at the heart
of the simulation.  Each module (only the crew module is shown above)
has a name and a list of resources that it consumes and produces along
with max and desired flow rates and what source(s) produce(s) that
resource (the {\tt inputs}) or from where the resource is drawn ({\tt
outputs}) More than one source or sink is allowed, but the module
tries to draw as much as it can from the first.  In the above snippet
some flow rates are 0 because the user has no control over them (e.g.,
crew air consumption is based on crew physiology and can't be directly
controlled).  As shown above, you can have as many (or as few) crew
members as you want of various ages, weights and sexes.  You can give
a default schedule for their day (which can be changed via a CORBA API
call during simulation runs).

The rest of the XML file specification proceeds in a similar way.  The
best way to change the configuration is to start with the default file
and edit it to reflect your requirements.  The default configuration
is based on BIO-Plex\index{BIO-Plex} \cite{tri99}, a NASA JSC ground
testbed for advanced life support.  As such, if all you want is a
``typical'' advanced life support configuration, then just use the
default XML file.

\subsection{Initializing BioSim}

Talk about alternative initialization\index{initialization} via an
API.  Again, capacities, levels, logging, etc. can be set from here.

\section{Enabling stochastic processes}
\index{stochastic}

BioSim provides the ability to change the stochastic nature of the
underlying modules via API calls or the XML initialization file.  If
you want a purely deterministic simulation you can do that.  If you
want a very unpredictable simulation you can do that as well.  This
section provides an overview of the stochastic controls available in
BioSim.  The type {\tt StochasticIntensity} includes the following
values:

\begin{itemize}
\item {\tt HIGH\_STOCH}
\item {\tt MEDIUM\_STOCH}
\item {\tt LOW\_STOCH}
\item {\tt NONE\_STOCH}
\end{itemize}

Each of these controls the width of the Gaussian\index{Gaussian}
function that provides the variations in output.  Setting the
stochastic intensity to NONE\_STOCH means that the simulation is
deterministic.  All noise is Gaussian.  The outputs of a module as
determined by the module's underlying equations are run through a
Gaussian filter before being output by the simulation.  The Gaussian
filter's parameters are controlled by the stochastic intensity, namely 
the higher the stochastic intensity, the higher the deviation.  

\subsection{Setting the stochastic intensity via the XML configuration file}

To change the stochastic intensity of a module in XML, simply add
setStochasticIntensity as an attribute to the module (if one isn't
listed, it defaults to NONE\_STOCH).  For example: 

\begin{verbatim}
<AirRS name="AirRS" setStochasticIntensity="HIGH_STOCH">
\end{verbatim}

The stochastic level for the entire simulation (which is equivalent to
setting each module's stochastic intensity) can by adding
setStochasticIntensity to the Globals section at the head of the XML
document.  For example:

\begin{verbatim}
<Globals crewsToWatch="CrewGroup" setStochasticIntensity="LOW_STOCH">
\end{verbatim}


\subsection{Setting the stochastic intensity via the API}
Each module of BioSim (i.e., the {\tt SimBioModule} class) has the
following methods:

\begin{itemize}
\item {\tt getStochasticIntensity()}: returns type {\tt
    StochasticIntensity}, which is the current stochastic level of the module.
\item {\tt setStochasticIntensity(StochasticIntensity pIntensity)}:
  Sets the current stochastic intensity of the module to {\tt
  pIntensity}.
\end{itemize}

The stochastic level for the entire simulation (which is equivalent to
setting each module's stochastic intensity) can be set via the {\tt
BioDriver} class using the {\tt setStochasticIntensity} method with
the same parameter ({\tt (StochasticIntensity})

\section{Logging}
\index{logging}

BioSim includes a facility for logging all data produced by the
simulation.  This facility can be used to debug your controller or to
create displays.  The data is saved as an XML file in
\verb+$BIOSIM_HOME/biosim-log.xml+ This section discussed controlling
data logging in BioSim.

\subsection{Setting logging via the XML configuration file}

Logging in BioSim is done through log4j, a popular logging utility for
Java.  By default, BioSim logs to the console at the "INFO" level.  With log4j,
output can be outputted to files, consoles, databases, sockets, etc.  Individual
components of the simulation can have their logging turned on.  For example, to turn
on debugging of air sensors, the following change must be added to the Globals section of
the BioSim XML configuration file: 
\begin{verbatim}
<log4jProperty name="log4j.logger.com.traclabs.biosim.server.sensor.air" value="DEBUG"/>
\end{verbatim}

You can also globally enable logging for the senors by adding:
\begin{verbatim}
<log4jProperty name="log4j.logger.com.traclabs.biosim.server.sensor" value="DEBUG"/>
\end{verbatim}
See log4j's documentation for more sophisticated logging options here:
\begin{verbatim}
http://logging.apache.org/log4j
\end{verbatim}
\subsection{Setting logging via the API}

Each module of BioSim (i.e., the {\tt SimBioModule} class) has the
following methods:

\begin{itemize}
\item {\tt isLogging()}: returns type {\tt
    boolean}, whether the module is logging.
\item {\tt setLogging(boolean pLogging)}:
  Sets whether the module should log.
\end{itemize}

\begin{figure}
\insertfigure{controller}{4.0in}
\caption{Interaction with the simulation.}
\label{fig:controller}
\end{figure}

\chapter{Controlling the simulation}
The previous two chapters showed you how to run BioSim, monitor its
internal state via a graphical user interface and configure it to your
specifications.  While interesting, this is not useful for control
system research.  This chapter will show you how to connect an
external control program to BioSim and change its underlying
operation.  Note that this chapter will not present an exhaustive list
of all of the objects and methods in BioSim, but only the most
important for new users.  Complete documentation of all objects and
method calls is available at:
\begin{verbatim}
  http://www.traclabs.com/biosim/doc/api/index.html
\end{verbatim}

Figure~\ref{fig:controller} shows the basic control strategy for the
simulation.  The simulation models represent the underlying equations
that govern the producer/consumer relationships (see
Section~\ref{sec:pc}).  Sensors and actuators are provided to access
the underlying simulations.  The user is responsible for writing the
controller, which uses the sensors and actuators to control the
simulation.  The user can also inject malfunctions into the underlying
simulation models and resupply consumables.   

\section{Controlling simulation runs}
The root class of the simulation is the {\tt BioDriver} class, which
contains the methods used to control the simulation.  The basic
control unit is a {\bf tick}\index{tick}, which advances the
simulation exactly one hour.  After creating your own BioDriver object
the
following methods are available:

\begin{enumerate}
\item  {\tt void advanceOneTick()}: Moves all modules of the
  simulation ahead exactly one hour.
\item {\tt void endSimulation()}: Ends\index{end} the simulation entirely.
\item {\tt boolean isDone()}: \index{isDone} Returns true if the simulation has met
  an end condition and has stopped.
\item {\tt void reset() }:\index{reset} Returns the simulation to its initial
  state.
\end{enumerate}

Here's an example of using one of these methods after a BioDriver
class has been created (called {\tt myBioDriver}) in a Java program:

\begin{verbatim}

    myBioDriver.advanceOneTick();

\end{verbatim}

There are a number of other methods available in the BioDriver class
that are less commonly used than those listed above.  These are
available via the JavaDoc\index{JavaDoc} facility.  After you have
installed your source code you can make documentation via the {\tt
  make-docs.sh} (for Linux) or {\tt make-docs} (for Windows)
executables in the {\tt bin} directory.  The BioDriverImpl class is
under the biosim.server.framework package.

\section{Accessing sensors and actuators}

\subsection{Generic sensors}

Sensors\index{sensors} are objects that connect to the underlying
simulation.  Each module has its own set of sensor objects, but they
all inherit the same method calls.  These method calls are:

\begin{itemize}

        \item {\tt getMax:} Returns the maximum allowed value of the sensor
        \item {\tt getMin:} Returns the minimum allowed value of the sensor
        \item {\tt getValue}: Returns the current setting of the sensor
\end{itemize}


\subsubsection{Additional sensor methods}

Several other generic methods related to sensor noise, malfunctions
and logging are available:

In addition, there will be some sensor-specific methods, which we will
discuss with the appropriate sensor.

\subsection{Generic actuators}

Actuators\index{actuators} are objects that connect to the underlying
simulation.  Each module has its own set of actuator objects, but they
all inherit the same method calls.  These method calls are:

\begin{itemize}

        \item {\tt getMax}: Returns the maximum allowed value of the actuator
        \item {\tt getMin}: Returns the minimum allowed value of the actuator
        \item {\tt getValue}: Returns the current value of the actuator       
        \item {\tt setValue}: Sets the actuator

\end{itemize}

\subsubsection{Additional actuator methods}

Several other generic methods related to actuator noise, malfunctions
and logging are available:

In addition, there will be some actuator-specific methods, which we
will discuss with the appropriate actuator.


\subsection{Environments}

Environments\index{environment} contain the air\index{air} that crew
and crops use.  As there can be multiple environments in the simulation,
there can be multiple sensors tied to each environment.

\subsubsection{Sensors}
\label{sec:env_sen}
The following sensors are available in the environment:

\begin{enumerate}

\item {\tt AirInFlowRateSensor}: Flow rate in moles/tick of air into the environment
\item {\tt AirOutFlowRateSensor}: Flow rate in moles/tick of air out of the environment
\item {\tt CO2AirConcentrationSensor}: Percentage of CO2 in the environment.
\item {\tt CO2AirStoreInFlowRateSensor}: Flow rate in liters/tick of
  CO2 from a store (paired with CO2\-Air\-Environment\-In\-Flow\-Rate\-Sensor).
\item {\tt CO2AirEnvironmentInFlowRateSensor}: Flow rate in liters/tick of CO2 into the environment (paired with CO2\-Air\-Store\-In\-FlowRate\-Sensor)
\item {\tt CO2AirPressureSensor}: Pressure in kPA of CO2 in the environment
\item {\tt CO2AirStoreOutFlowRateSensor}: Flow rate in liters/tick of CO2 from a store. (paired with CO2\-Air\-Environment\-Out\-FlowRate\-Sensor)
\item {\tt CO2AirEnvironmentOutFlowRateSensor}: Flow rate in liters/tick of CO2 out of the environment (paired with CO2\-Air\-Store\-Out\-FlowRate\-Sensor)
\item {\tt NitrogenAirConcentrationSensor}: Percentage of Nitrogen in the environment
\item {\tt NitrogenAirEnvironmentInFlowRateSensor}: Flow rate in liters/tick of nitrogen into the environment (paired with Nitrogen\-Air\-Store\-In\-Flow\-Rate\-Sensor)
\item {\tt NitrogenAirStoreInFlowRateSensor}: Flow rate in liters/tick of nitrogen into a store (paired with Nitrogen\-Air\-Environment\-In\-Flow\-Rate\-Sensor)
\item {\tt NitrogenAirEnvironmentOutFlowRateSensor}: Flow rate in liters/tick of nitrogen out of an environment (paired with Nitrogen\-Air\-Store\-Out\-Flow\-Rate\-Sensor)
\item {\tt NitrogenAirStoreOutFlowRateSensor}: Flow rate in liters/tick of nitrogen out of a store (paired with Nitrogen\-Air\-Environment\-Out\-Flow\-Rate\-Sensor)
\item {\tt NitrogenAirPressureSensor}: Pressure in kPA of nitrogen in the environment
\item {\tt O2AirConcentrationSensor}: Percentage of O2 in the environment.
\item {\tt O2AirEnvironmentInFlowRateSensor}: Flow rate in liters/tick of O2 into the environment (paired with O2\-Air\-Store\-In\-Flow\-Rate\-Sensor)
\item {\tt O2AirStoreInFlowRateSensor}: Flow rate in liters/tick of O2 out of the store (paired with O2\-Air\-Environment\-In\-Flow\-Rate\-Sensor)
\item {\tt O2AirEnvironmentOutFlowRateSensor}: Flow rate in liters/tick of O2 out of the environment (paired with O2\-Air\-Store\-Out\-Flow\-Rate\-Sensor)
\item {\tt O2AirStoreOutFlowRateSensor}: Flow rate in liters/tick of O2 out of the store (paired with O2\-Air\-Environment\-Out\-Flow\-Rate\-Sensor)
\item {\tt O2AirPressureSensor}: Pressure in kPA of O2 in the environment
\item {\tt OtherAirConcentrationSensor}: Percentage of trace gases in the environment.
\item {\tt OtherAirPressureSensor}: Pressure in kPA of trace gases in the environment
\item {\tt WaterAirConcentrationSensor}: Percentage of water (humidity) in the environment
\item {\tt WaterAirEnvironmentInFlowRateSensor}: Flow rate in liters/tick of water vapor into the environment (paired with Water\-Air\-Store\-In\-Flow\-Rate\-Sensor)
\item {\tt WaterAirStoreInFlowRateSensor}: Flow rate in liters/tick of water vapor into the store (paired with Water\-Air\-Environment\-In\-Flow\-Rate\-Sensor)
\item {\tt WaterAirEnvironmentOutFlowRateSensor}: Flow rate in liters/tick of water vapor out of the environment (paired with Water\-Air\-Store\-Out\-Flow\-Rate\-Sensor)
\item {\tt WaterAirStoreOutFlowRateSensor}: Flow rate in liters/tick of water vapor out of the store (paired with Water\-Air\-Environment\-Out\-Flow\-Rate\-Sensor)
\item {\tt WaterAirPressureSensor}: Pressure in kPA of water in the environment
\end{enumerate}

\subsubsection{Actuators}

\begin{enumerate}

\item {\tt AirInFlowRateActuator}: Flow rate in moles/tick of air into the environment
\item {\tt AirOutFlowRateActuator}: Flow rate in moles/tick of air out of the environment
\item {\tt CO2AirStoreInFlowRateActuator}: Flow rate in liters/tick of
  CO2 from a store (paired with CO2\-Air\-Environment\-In\-FlowRate\-Actuator)
\item {\tt CO2AirEnvironmentInFlowRateActuator}: Flow rate in liters/tick of CO2 into the environment (paired with CO2\-AirStore\-In\-FlowRate\-Actuator)
\item {\tt CO2AirStoreOutFlowRateActuator}: Flow rate in liters/tick of CO2 from a store. (paired with CO2\-Air\-Environment\-Out\-Flow\-Rate\-Actuator)
\item {\tt CO2AirEnvironmentOutFlowRateActuator}: Flow rate in liters/tick of CO2 out of the environment (paired with CO2AirStoreOutFlowRateActuator)
\item {\tt NitrogenAirEnvironmentInFlowRateActuator}: Flow rate in liters/tick of nitrogen into the environment (paired with Nitrogen\-Air\-Store\-In\-Flow\-Rate\-Actuator)
\item {\tt NitrogenAirStoreInFlowRateActuator}: Flow rate in liters/tick of nitrogen into a store (paired with Nitrogen\-Air\-Environment\-In\-Flow\-Rate\-Actuator)
\item {\tt NitrogenAirEnvironmentOutFlowRateActuator}: Flow rate in liters/tick of nitrogen out of an environment (paired with Nitrogen\-Air\-Store\-Out\-Flow\-Rate\-Actuator)
\item {\tt NitrogenAirStoreOutFlowRateActuator}: Flow rate in liters/tick of nitrogen out of a store (paired with Nitrogen\-Air\-Environment\-Out\-Flow\-Rate\-Actuator)
\item {\tt O2AirEnvironmentInFlowRateActuator}: Flow rate in liters/tick of O2 into the environment (paired with O2\-Air\-Store\-In\-Flow\-Rate\-Actuator)
\item {\tt O2AirStoreInFlowRateActuator}: Flow rate in liters/tick of O2 out of the store (paired with O2\-Air\-Environment\-In\-Flow\-Rate\-Actuator)
\item {\tt O2AirEnvironmentOutFlowRateActuator}: Flow rate in liters/tick of O2 out of the environment (paired with O2\-Air\-Store\-Out\-Flow\-Rate\-Actuator)
\item {\tt O2AirStoreOutFlowRateActuator}: Flow rate in liters/tick of O2 out of the store (paired with O2\-Air\-Environment\-Out\-Flow\-Rate\-Actuator)
\item {\tt WaterAirEnvironmentInFlowRateActuator}: Flow rate in liters/tick of water vapor into the environment (paired with Water\-Air\-Store\-In\-Flow\-Rate\-Actuator)
\item {\tt WaterAirStoreInFlowRateActuator}: Flow rate in liters/tick of water vapor into the store (paired with Water\-Air\-Environment\-In\-Flow\-Rate\-Actuator)
\item {\tt WaterAirEnvironmentOutFlowRateActuator}: Flow rate in liters/tick of water vapor out of the environment (paired with Water\-Air\-Store\-Out\-Flow\-Rate\-Actuator)
\item {\tt WaterAirStoreOutFlowRateActuator}: Flow rate in liters/tick of water vapor out of the store (paired with Water\-Air\-Environment\-Out\-Flow\-Rate\-Actuator)

\end{enumerate}

\subsubsection{Malfunctions}
State exactly what the malfunctions do.  How do the method calls work.
What is the malfunction. What is the leak rate, etc. for each of the
different malfunction classes.  Since method calls are probably
identical may want to move this up to top and then use this section
only for module-specific malfunction information.

\subsection{Crew}
\index{crew}
\subsubsection{Sensors}

\begin{enumerate}

\item {\tt CrewGroupActivitySensor}:  Returns average activity level of crew
\item {\tt CrewGroupAnyDeadSensor}: Returns whether any crew member has died
\item {\tt CrewGroupDeathSensor}: Return whether the entire crew has died
\item {\tt CrewGroupProductivitySensor}: Total productivity of the crew

\end{enumerate}

\subsubsection{Actuators}

\begin{enumerate}

\item {\tt CrewGroupActivityActuator}:

\end{enumerate}

\subsubsection{Malfunctions}

\subsection{Air}
\index{air}
\subsubsection{Sensors}

\begin{enumerate}
\item {\tt CO2InFlowRateSensor}: Reports the CO2 flow rate (in moles) into a module
\item {\tt CO2OutFlowRateSensor}: Reports the CO2 flow rate (in moles) out of a module
\item {\tt CO2StoreLevelSensor}: How much CO2 (in moles) is contained in this store.
\item {\tt H2InFlowRateSensor}: Reports the H2 flow rate (in moles) into a module
\item {\tt H2OutFlowRateSensor}: Reports the H2 flow rate (in moles) out of a module
\item {\tt H2StoreLevelSensor}: How much   (in moles) is contained in this store.
\item {\tt NitrogenInFlowRateSensor}: Reports the nitrogen flow rate (in moles) into a module
\item {\tt NitrogenOutFlowRateSensor}: Reports the nitrogen flow rate (in moles) out of a module
\item {\tt NitrogenStoreLevelSensor}: How much nitrogen (in moles) is contained in this store.
\item {\tt O2InFlowRateSensor}: Reports the O2 flow rate (in moles) into a module
\item {\tt O2OutFlowRateSensor}: Reports the O2 flow rate (in moles) out of a module
\item {\tt O2StoreLevelSensor}: How much O2 (in moles) is contained in this store.
\end{enumerate}

\subsubsection{Actuators}

\begin{enumerate}
\item {\tt CO2InFlowRateAcutator}: Changes how much CO2 (in moles) goes into the module
\item {\tt CO2OutFlowRateAcutator}: Changes how much CO2 (in moles) goes out of the module
\item {\tt H2InFlowRateAcutator}: Changes how much H2 (in moles) goes into the module
\item {\tt H2OutFlowRateAcutator}: Changes how much H2 (in moles) goes out of the module
\item {\tt NitrogenInFlowRateAcutator}: Changes how much nitrogen (in moles) goes into the module
\item {\tt NitrogenOutFlowRateAcutator}: Changes how much nitrogen (in moles) goes out of the module
\item {\tt O2InFlowRateAcutator}: Changes how much O2 (in moles) goes into the module
\item {\tt O2OutFlowRateAcutator}: Changes how much O2 (in moles) goes out of the module
\end{enumerate}

\subsubsection{Malfunctions}

Same as others


\subsection{Water}
\index{water}

\subsubsection{Sensors}
\begin{enumerate}
\item {\tt DirtyWaterInFlowRateSensor}: Reports the dirty water flow rate (in liters) into a module
\item {\tt DirtyWaterOutFlowRateSensor}: Reports the dirty water flow rate (in liters) out of a module
\item {\tt DirtyWaterStoreLevelSensor}: How much dirty water (in liters) is contained in this store.
\item {\tt GreyWaterInFlowRateSensor}: Reports the grey water flow rate (in liters) into a module
\item {\tt GreyWaterOutFlowRateSensor}: Reports the grey water flow rate (in liters) out of a module
\item {\tt GreyWaterStoreLevelSensor}: How much grey water (in liters) is contained in this store.
\item {\tt PotableWaterInFlowRateSensor}: Reports the potable water flow rate (in liters) into a module
\item {\tt PotableWaterOutFlowRateSensor}: Reports the potable water flow rate (in liters) out of a module
\item {\tt PotableWaterStoreLevelSensor}: How much potable water (in liters) is contained in this store.
\end{enumerate}

\subsubsection{Actuators}

\begin{enumerate}
\item {\tt DirtyWaterInFlowRateActuator}: Changes how much dirty water (in liters) goes into the module
\item {\tt DirtyWaterOutFlowRateActuator}: Changes how much dirty water (in liters) goes ouf of the module
\item {\tt GreyWaterInFlowRateActuator}: Changes how much grey water (in liters) goes into the module
\item {\tt GreyWaterOutFlowRateActuator}: Changes how much grey water (in liters) goes ouf of the module
\item {\tt PotableWaterInFlowRateActuator}: Changes how much potable water (in liters) goes into the module
\item {\tt PotableWaterOutFlowRateActuator}: Changes how much potable water (in liters) goes ouf of the module
\end{enumerate}

\subsubsection{Malfunctions}

\subsection{Biomass}
\index{biomass}

\subsubsection{Sensors}
\begin{enumerate}
\item {\tt BiomassInFlowRateSensor}: Reports the biomass flow rate (in kilograms) into a module
\item {\tt BiomassOutFlowRateSensor}: Reports the biomass flow rate (in kilograms) out of a module
\item {\tt BiomassStoreLevelSensor}: How much biomass (in kilograms) is contained in this store.
\item {\tt FoodInFlowRateSensor}: Reports the food flow rate (in kilograms) into a module
\item {\tt FoodOutFlowRateSensor}: Reports the food flow rate (in kilograms) out of a module
\item {\tt FoodStoreLevelSensor}: How much food (in kilograms) is contained in this store.
\end{enumerate}

\subsubsection{Actuators}
\begin{enumerate}
\item {\tt BiomassInFlowRateActuator}: Changes how much biomass (in kilograms) goes into the module
\item {\tt BiomassOutFlowRateActuator}: Changes how much biomass (in kilograms) goes out of the module
\item {\tt FoodInFlowRateActuator}: Changes how much food (in kilograms) goes into the module
\item {\tt FoodOutFlowRateActuator}: Changes how much food (in kilograms) goes out of the module
\end{enumerate}

\subsubsection{Malfunctions}

\subsection{Dry Waste}
\index{waste}
\subsubsection{Sensors}
\begin{enumerate}
\item {\tt DryWasteInFlowRateSensor}: Changes how much dry Waste (in kilograms) goes into the module
\item {\tt DryWasteOutFlowRateSensor}: Changes how much dry Waste (in kilograms) goes out of the module
\item {\tt DryWasteStoreLevelSensor}: How much dry Waste (in kilograms) is contained in this store.
\end{enumerate}

\subsubsection{Actuators}
\begin{enumerate}
\item {\tt DryWasteInFlowRateActuator}: Changes how much dry waste (in kilograms) goes into the module
\item {\tt DryWasteOutFlowRateActuator}: Changes how much dry waste (in kilograms) goes out of the the module
\end{enumerate}

\subsubsection{Malfunctions}

\subsection{Power}
\index{power}

\subsubsection{Sensors}

\begin{enumerate}
\item {\tt PowerInFlowRateSensor}: Changes how much power (in watts) goes into the module
\item {\tt PowerOutFlowRateSensor}: Changes how much power (in watts) goes out of the module
\item {\tt PowerStoreLevelSensor}: How much power (in watts) is contained in this store.
\end{enumerate}

\subsubsection{Actuators}

\begin{enumerate}
\item {\tt PowerInFlowRateActuator}: Changes how much power (in watts) goes into the module
\item {\tt PowerOutFlowRateActuator}: Changes how much power (in watts) goes out of the module
\end{enumerate}

\subsubsection{Malfunctions}



\chapter{Writing a simple Java controller}

This Java example uses the JacOrb to connect to the BioSim.  You can
cut and paste this code and follow the compiling and running
instructions.  This code is also included in source code distributed
with BioSim.  

To compile this code you need to do the following:

\begin{verbatim}

    % run make-client.sh
    % javac -classpath /
            .:$BIOSIM_HOME\lib\jacorb\jacorb.jar:$BIOSIM_HOME\generated\client\classes /
            TestBiosim.java 
\end{verbatim}

\noindent where {\tt javac} is the Java compiler, {\tt jacorb.jar} is
the library that has the ORB and various CORBA utilities, {\tt
  generated\\client\\classes} are the client stubs generated from the
IDL, and {\tt TestBiosim} is the file described in this chapter.

To run this code you will need to do the following:

\begin{verbatim}

    % run run-nameserver.sh
    % run run-server.sh
    % java -classpath /
           .:$BIOSIM_HOME\lib\jacorb\jacorb.jar:$BIOSIM_HOME\lib\jacorb:$BIOSIM_HOME\generated\client\classes /
           -Dorg.omg.CORBA.ORBClass=org.jacorb.orb.ORB /
           -Dorg.omg.CORBA.ORBSingletonClass=org.jacorb.orb.ORBSingleton /
           -DORBInitRef.NameService=file:$BIOSIM_HOME/generated/ns/ior.txt TestBiosim
\end{verbatim}

/noindent {\tt run-nameserver.sh} and \tt{run-server.sh} are in the
\verb+$BIOSIM_HOME/bin+ directory and start the CORBA name server and
the BioSim server respectively.  The {\tt -D} options tell Java that
we are using JacOrb and where to look for the ior file that the
nameserver creates.  

The code to connect to the BioSim server is as follows:

\begin{verbatim}

public class TestBiosim{
	public static void main(String[] args){
		try{
			System.out.println("TestBiosim begin");
			//The ORB
			ORB myOrb = null;
			//The naming context reference
			NamingContextExt myNamingContextExt = null;
			// create and initialize the ORB (with no arguments)
			String[] nullArgs = null;
			System.out.println("Initializing ORB");
			myOrb = ORB.init(nullArgs, null);
			System.out.println("Getting a reference to the naming service");
			org.omg.CORBA.Object nameServiceObject = myOrb.resolve_initial_references("NameService");
			System.out.println("Narrowing name service");
			myNamingContextExt = NamingContextExtHelper.narrow(nameServiceObject);

			//Let's get the BioDriver
			System.out.println("Getting BioDriver");
			BioDriver myBioDriver = BioDriverHelper.narrow(myNamingContextExt.resolve_str("BioDriver0"));
			System.out.println("Invoking method on BioDriver");
			//Now let's call a method on BioDriver
			myBioDriver.spawnSimulationAndRunTillDead();
			System.out.println("Invoking another method on BioDriver");
			//Now let's call another method on BioDriver, this time with a result
			int numberOfTicks = myBioDriver.getTicks();
			System.out.println("Result was: "+numberOfTicks);
			//All done!
			System.out.println("TestBiosim end");
		}
		catch (Exception e){
			System.out.println("Something went wrong!");
			e.printStackTrace();
		}
	}
}

\end{verbatim}


\chapter{Writing a simple C++ controller}
This C++ example uses the ACEOrb to connect to the BioSim server.  You
will need to install the ACEOrb on your machine.  The example also
assumes a Linux operating system using the g++ compiler.  

\section{Makefile}

We will use a Makefile to compile the C++ executable.  The Makefile
consists of the following:

\begin{verbatim}
INCLUDES = -I. -I$(TAO_ROOT) -I$(ACE_ROOT) -I$(TAO_ROOT)/orbsvcs/orbsvcs

biosim_client: biosimC.cpp biosim_client.cpp 
	       g++ -o biosim_client biosimC.cpp biosim_client.cpp /
                $(INCLUDES) /
	       -L$(TAO_ROOT)/ACE_wrappers/TAO/tao  /
               -L$(TAO_ROOT)/ACE_wrappers/TAO/orbsvcs/orbsvcs /
               -L$(TAO_ROOT)/ACE_wrappers/TAO/tao/PortableServer /	       
               -lTAO -lTAO_PortableServer -lTAO_CosNaming /
\end{verbatim}

\noindent Where {\tt TAO\_ROOT} and {\tt ACE\_ROOT} are environment
variables that contain the pathname where TAO and ACE are installed on
your machine and where {\tt biosimC.cpp} is a file generated from the
IDL and {\tt biosim\_client.cpp} is the code from this chapter.

\section{Client code}

\begin{verbatim}

#include "biosimC.h"
#include <CosNamingC.h>

static const char* CorbaServerName = "BioDriver0";
int
main(int argc,char *argv[])
{
  try {
    // Initialize orb
    CORBA::ORB_var orb = CORBA::ORB_init(argc,argv,"TestClient");

    cerr << "initialized orb" << endl;

    if (argc < 2) {
      cerr << "Usage: " << argv[0]
           << " IOR_string" << endl;
      return 1;
    }

// Use the first argument to create the object reference,
    // in real applications we use the naming service, but let's do
    // the easy part first!

    CORBA::Object_var obj =
      orb->string_to_object (argv[1]); 

    cout << "After resolve_initial_references" << endl;


    /*   CORBA::Object_var obj =
         orb->string_to_object (argv[1]); */
 
    CosNaming::NamingContext_var inc = CosNaming::NamingContext::_narrow(obj);

    cout << "After NameService narrow" << endl;

    if(CORBA::is_nil(inc)) {
      cerr << "Unable to narrow NameService." << endl;
      throw 0;
    }  // end if couldn't get root naming context
    
    CosNaming::Name name;
    name.length(1);
    name[0].id = CORBA::string_dup(CorbaServerName);
    name[0].kind = CORBA::string_dup("");

    CORBA::Object_var tmp = inc->resolve(name);

    cout << "After resolve name" << endl;

    if(CORBA::is_nil(tmp)) {
      cerr << "Unable to resolve " << CorbaServerName << " from NameService" << endl;
      throw 0;
    }  // end if couldn't find object

    cerr << "got object" << endl;

    // Now downcast the object reference to the appropriate type

    biosim::idl::framework::BioDriver_var myBiodriver = biosim::idl::framework::BioDriver::_narrow(tmp);


    cerr << "narrowed" << endl;

    // example of a call to the simulation

    myBiodriver->spawnSimulationTillDead();

    cerr << "spawned" << endl;

    orb->destroy ();
  }
  catch (CORBA::Exception &ex) {
    std::cerr << "CORBA exception raised! " << std::endl;
  }
  return 0;
}

\end{verbatim}

\section{Running the client code}
To run this code you will need to do the following:

\begin{verbatim}

    % run run-nameserver
    % run run-server
    % ./biosim_client
\end{verbatim}

/noindent {\tt run-nameserver.sh} and \tt{run-server.sh} are in the
\verb+$BIOSIM_HOME/bin+ directory and start the CORBA name server and
the BioSim server respectively.

\chapter{Control Examples}

\section{JSC genetic algorithm}

\section{Rice reinforcement learner}

\section{Texas Tech reinforcement learner}

\chapter{Conclusions}

\bibliography{biosim}
\bibliographystyle{plain}

\begin{theindex}

\documentstyle{report}

\input epsf

%\nofiles

\setlength{\textheight}{9.0in}
\setlength{\textwidth}{6.5in}
\setlength{\footheight}{0.0in}
\setlength{\topmargin}{0.00in}
\setlength{\headheight}{0.0in}
\setlength{\headsep}{0.0in}
\setlength{\oddsidemargin}{-0.0in}

\makeindex

\begin{document}

\def\topfraction{1.0}        %maximum fraction of floats at the top of the page
\def\bottomfraction{1.0}     %ditto, for the bottom of the page
\def\textfraction{0}         %minimum fraction of text (--> 100% floats is okay)
\def\floatpagefraction{1.0}  % if a page is full of floats, it'd better be FULL

\newcommand{\insertfigure}[2]{\begin{center}
                             \ \epsfxsize=#2 \epsfbox{#1.ps}
                             \end{center}}

%\newcommand{\insertfigure}[2]{\vspace{#2}}

\title{BioSim: An Integrated Simulation of an Advanced Life Support System for Intelligent Control Research\\
Users Manual (draft)}

\author{Metrica Inc. and S\&K Technologies\\
NASA Johnson Space Center/ER2\\
Houston TX 77058\\
http://www.traclabs.com/biosim}


\date{\today}

\maketitle

\tableofcontents

\newpage

\listoffigures

\newpage

\chapter{Introduction}
Advanced life support systems have multiple interacting subsystems,
which makes control a particularly challenging task.  The simulation
described in this document provides a testbed for integrated control
research.  There have been other integrated life support simulations
(e.g., \cite{finn99}) and we have learned from those efforts.  This
simulation is designed exclusively for integrated controls research,
which imposes different requirements.  For example, the simulation is
accessed through sensors and actuators, just as a real system would
be.  Noise and uncertainty are built in and controllable.
Malfunctions and failures of subsystems are modeled and manifest
themselves through anomalous readings in the sensors.  Crew members
are taskable and their tasks have purpose and meaning in the
simulation.  In essence, the simulation is a replacement for the
Advanced Life Support (ALS) hardware and crew, allowing for testing of
control approaches in advance of any integrated test.  We want the
simulation to be used to develop and evaluate integrated control
techniques.  There are still many open research questions with respect
to controlling advanced life support systems.  For example is a
distributed or hierarchical approach better?  What role does machine
learning play in control?  How can symbolic, qualitative control
approaches be integrated with continuous, quantitative approaches?
How can we evaluate different control philosophies?  We hope that the
controls community can use this simulation to begin to build systems
which will provide us with answers to these kinds of questions. 

The simulation is written in Java and is accessed using the Common
Object Request Broker Architecture (CORBA)\index{CORBA}.  The distributed nature of
the simulation allows for multiple instances to be run in parallel,
making it ideal for testing advanced control concepts such as genetic
algorithms or reinforcement learning, which require multiple trials.
See \cite{kortenkamp_etal01} or \cite{schreckenghost_etal02} for a
background on advanced life support control issues and previous work.
This document begins with a brief introduction to advanced life
support systems and provides details on installing, running and
interacting with the simulation.  The simulation is available from:

\begin{verbatim}
     http://www.traclabs.com/biosim
\end{verbatim}

\chapter{Background on advanced life support}

An advanced life support system is one in which many resources are
reused or regenerated.  Such systems will be necessary for long
duration human missions, such as those to Mars, where resupply
opportunities are limited.  Typically they have thin margins for mass,
power and buffers, which requires optimization and tight control.
Also, advanced life support systems consist of many interconnected
subsystems all of which interact in both predictable and unpredictable
ways (see Figure~\ref{fig:biosim}).  Autonomous control\index{control}
of these systems is desirable. This section gives enough background so
that users of the simulation can understand the major components of an
advanced life support system.  For detailed documentation on advanced
life support systems see \cite{tri99} or go to:

\begin{verbatim}
    http://advlifesupport.jsc.nasa.gov
\end{verbatim}

\begin{figure}
\insertfigure{figure}{5.5in}
\caption{The various modules that comprise the BIOSim integrated simulation.}
\label{fig:biosim}
\end{figure}

\section{Modules}

An advanced life support system consists of many interacting
subsystems.  While each is self-contained, they rely on each other for
various resources.  The simulation is built from a set of modules that
each produce and consume different resources.

\subsection{Environment}
\label{sec:env}
An environment\index{environment} contains air that is consumed by
either people or crops.  Air\index{air} contains a mixture of gases --
in our simulation these gases are oxygen (O2), carbon dioxide (CO2),
nitrogen (N), water vapor (H20), and other gasses (trace).  The initial composition of the
gases is set by the simulation initialization file (see
Section~\ref{sec:init}).  As the simulation runs modules may consume
air from the simulation and replace it with air of a different
composition.  Thus, the composition of gases and pressure in the air change over
time and can be measured by environment sensors
(Section~\ref{sec:env_sen}).  As with all modules, there can be
multiple environments.  For example, it is common for crew members
and crops to have different air compositions and that is the default
in this simulation.   

\subsection{Crew}
\label{sec:crew}

The crew\index{crew} module is implemented using models described in
\cite{goudarzi_ting99}.  The number, gender, age and weight of the
crew are settable as input parameters -- in the default configuration
there are four crew members, two male and two female.  The crew cycles
through a set of activities (sleep, maintenance, recreation, etc.).
As they do so they consume O2, food and water and produce CO2, dirty
water and solid waste.  The amount of resources consumed and produced
varies according to crew member attributes and their activities.  The
crew's activities can be adjusted by passing a new crew schedule to
the crew module.  A default schedule can also be used.  The crew
module is connected to a crew environment that contains an atmosphere
that they breathe.  The initial size and gas composition (percentages
of O2, CO2, H2O and inert gases) are input parameters and the default
is $1.54893 x 10^6 liters$ (from \cite{tri99}) with an atmosphere
equivalent to sea level air.  As the simulation progresses the mixture
of gases in the atmosphere changes.

\subsection{Water}
\label{sec:water}
Consumes:Power, Grey Water, Dirty Water
Produces: Potable Water

The water\index{water} recovery module consumes dirty water, grey water(i.e., water that can be used for
washing but not drinking), and power and
produces potable water.  The water recovery module consists of four
subsystems that process the water.  The biological water processing
(BWP) subsystem removes organic compounds.  Then the water passes to a
reverse osmosis (RO) subsystem, which makes 85\% of the water passing
through it grey.  The 15\% of water remaining from the RO (called
brine) is passed to the air evaporation subsystem (AES), which
recovers the rest.  These two streams of grey water (from the RO and
the AES) are passed through a post-processing subsystem (PPS) to
create potable water.  An external controller can turn on or off
various subsystems.  For example, all water can pass through the AES
at a higher power cost.  We based our water recovery module on a
recently completed test at NASA Johnson Space Center
\cite{bonasso_etal03}.

\subsection{Air}
\label{sec:air}
Consumes:Power, H2, Air, CO2, Potable Water
Produces: Air (with less CO2), O2, H2, CO2, Potable Water

The air\index{air} component takes in exhalant CO2 and produces O2 as long as
there is sufficient power being provided to the system.  This
component is modeled on various Air Revitalization System (ARS) work
at NASA JSC \cite{malin_etal00}.  There are three interacting air
subsystems: the Variable Configuration Carbon Dioxide Removal (VCCR)
System in which CO2 is removed from the air stream; the Carbon Dioxide
Reduction System (CRS), which also removes CO2 from the air stream
using a different process and producing different gases than the VCCR;
and the Oxygen Generation System (OGS) in which O2 is added to the air
stream by breaking water down into hydrogen and oxygen.  It is
important to note that both the removal of CO2 and the addition of O2
are required for human survival.  It is also important to note that
the biomass component (next subsection) also removes CO2 and adds O2.

\subsection{Biomass}
\label{sec:biomass}
Consumes:Power, Potable Water, Grey Water, Air
Produces: Air (with more CO2), Biomass, Dirty Water, CO2, Potable Water

The biomass\index{biomass} component is where crops\index{crops} are
grown.  It produces both biomass, which can be turned into
food\index{food}, and O2 and consumes water, power (light) and CO2.
This component is based on models given in \cite{jones_cavazzoni00}.
The system is modeled as shelves\index{shelves} that contain plants,
lights and water.  Shelves are planted and harvested and there is
growth cycle for each shelf.  Currently, ten crops are modeled and can
be planted in any ratio.

\subsection{Food processing}
\label{sec:food}
Consumes:Power, Biomass
Produces: Food

Before biomass can be consumed by the crew it must be converted to
food\index{food}.  The food processing component takes biomass, power and crew
time and produces food and solid waste.  The crew needs to be involved
in this process as it is labor intensive.  See Section~\ref{sec:crew}
for more information on scheduling crew activities.

\subsection{Waste}
\label{sec:waste}
Consumes:Power, Dry Waste, O2
Produces: CO2

The waste\index{waste} component consumes power, O2 and solid waste
and produces CO2.  It is modeled on an incinerator used in the Phase
III test in 1997 \cite{schreckenghost_etal02}.  Incineration can be
scheduled.

\subsection{Thermal control}
\label{sec:thermal}
Consumes: ?
Produces: ?

The thermal control\index{thermal} component regulates the air temperature in the
habitat.  This component is not implemented for this simulation and it
is assumed that external controllers are maintaining chamber
temperature.

\subsection{Power}
\label{sec:power}
Consumes: 
Produces: Power

The power\index{power} component supplies power to all of the other
components that need it.  There are two choices for power in the
simulation.  The first is nuclear\index{nuclear} power, which supplies
a fixed amount throughout the lifetime of the simulation.  The second
is solar\index{solar} power, which supplies a varying amount
(day/night cycle) of power to each component.

\subsection{Accumulators}
\label{sec:accumulator}
Consumes: All
Produces: All

The accumulator\index{accumulator} component can take a resource from any
store or environment and place it into another environment or store.  It is
functionally equivalent to an injector.

\subsection{Injectors}
\label{sec:injector}
Consumes: All
Produces: All

The injector\index{injector} component can take a resource from any
store or environment and place it into another environment or store.  It is
functionally equivalent to an accumulator.

\chapter{Simulation Properties}

The simulation implements the modules outlined in the previous
section.  BioSim does {\em not} simulate at the level of valves,
pumps, switches, etc.  Instead, modules are implemented in a
producer/consumer relationship, which is described in the next
subsection.  The simulation also provides additional functionality to
help test and debug control programs.

\section{Producer/Consumer Model}

No component in BioSim directly interacts with other components.  Instead, each component
has a rigid set of resources that it consumes and produces.  The resources are taken from stores/environments
and put into stores/environments.  The resources for BioSim currently consist of
power, potable water, grey water, dirty water, air, H2, nitrogen, O2, CO2, biomass, food, and dry waste.  At each simulation tick,
each module takes resource from it's store (it's consumables), and puts resources into stores (it's products).  Stores always report
thier values (level and capacity) from the last tick and report new values after every component in the simulation has been ticked.  
Resource conflicts can occur when two components ask for a limited resource, i.e. two WaterRS's both need 20 liters of dirty water and there
is only 20 liters in the dirty water tank.  This is currently resolved on a winner-take-all basis that is for all practical purposes
randomized.

\section{Success criteria}

In order to compare different control approaches there should be
objective criteria for a successful advanced life support system
mission.  Several possibilities exist already in the simulation.  For
example, the length of the mission before consumables are gone is a
success criterion.  As is minimizing the starting levels of stores or
minimizing the sizes of intermediary buffers.  However, many of these
fail to get at the true success of a mission, the productivity of the
crew in performing their science objectives.  Thus, the simulation
includes an artificial productivity\index{productivity} measure.  As
the crew cycles through their activities (see Section~\ref{sec:activities})
the amount of time they spend doing "mission" tasks is accumulated. This
number is also multiplied by a factor that takes into account the
amount of sleep, exercise, water, food and oxygen they are getting to approximate 
crew effectivness at performing the task (a happy crew is a productive crew!).
So each tick, for each crew member the productivity is given by the
following equation:

        [need to put the equation here]
	[some relevant code, derive equation later]
	\begin{verbatim}
	private static final float WATER_TILL_DEAD = 8.1f;
	private static final float WATER_RECOVERY_RATE=0.01f;
	private static final float OXYGEN_TILL_DEAD = 3f;
	private static final float OXYGEN_RECOVERY_RATE=0.01f;
	private static final float CALORIE_TILL_DEAD = 180000f;
	private static final float CALORIE_RECOVERY_RATE=0.0001f;
	private static final float DANGEROUS_CO2_RATION = 0.06f;
	private static final float CO2_TILL_DEAD = 10f;
	private static final float CO2_RECOVERY_RATE=0.001f;
	private static final float LEISURE_TILL_BURNOUT = 168f;
	private static final float LEISURE_RECOVERY_RATE=90f;
	private static final float AWAKE_TILL_EXHAUSTION = 120f;
	private static final float SLEEP_RECOVERY_RATE=12f;
	consumedWaterBuffer = new SimpleBuffer(WATER_TILL_DEAD, WATER_TILL_DEAD);
	consumedOxygenBuffer = new SimpleBuffer(OXYGEN_TILL_DEAD, OXYGEN_TILL_DEAD);
	consumedCaloriesBuffer = new SimpleBuffer(CALORIE_TILL_DEAD, CALORIE_TILL_DEAD);
	consumedCO2Buffer = new SimpleBuffer(CO2_TILL_DEAD, CO2_TILL_DEAD);
	sleepBuffer = new SimpleBuffer(AWAKE_TILL_EXHAUSTION, AWAKE_TILL_EXHAUSTION);
	leisureBuffer = new SimpleBuffer(LEISURE_TILL_BURNOUT, LEISURE_TILL_BURNOUT);
	float caloriePercentFull = sigmoidLikeProbability(consumedCaloriesBuffer.getLevel() / consumedCaloriesBuffer.getCapacity());
	float waterPercentFull = sigmoidLikeProbability(consumedWaterBuffer.getLevel() / consumedWaterBuffer.getCapacity());
	float oxygenPercentFull = sigmoidLikeProbability(consumedOxygenBuffer.getLevel() / consumedOxygenBuffer.getCapacity());
	float CO2PercentFull = sigmoidLikeProbability(consumedCO2Buffer.getLevel() / consumedCO2Buffer.getCapacity());
	float sleepPercentFull = sleepBuffer.getLevel() / sleepBuffer.getCapacity();
	float leisurePercentFull = leisureBuffer.getLevel() / leisureBuffer.getCapacity();
	float averagePercentFull = (caloriePercentFull + waterPercentFull + oxygenPercentFull + CO2PercentFull + sleepPercentFull + leisurePercentFull) / 6f;
	myMissionProductivity += myCrewGroup.randomFilter(averagePercentFull);
	\end{verbatim}
\section{Stochastic processes}

Any sufficiently complex process, especially one with biological
components, will not be deterministic\index{deterministic}.  That is,
given the same starting conditions and inputs it will not produce the
exact same outputs each time it is run.  A
stochastic\index{stochastic} process is one in which chance or
probability affect the outcome.  This simulation offers both
deterministic and stochastic operations.  In the former case, running
the simulation twice with the same inputs and initial conditions will
produce the same results.  This may be useful for quantitatively
comparing different control approaches against each other.  In the
latter case, the user can control the amount of variability in the
simulation (see Section 7.4).  This is modeled using a Gaussian\index{Gaussian}
distribution.  The deviation is determined by the stochastic intensity; a higher intensity will yield a higher deviation.  The filter is then appropriately applied to certain variable of the model.  This can be used to test a
control system�s ability to deal with stochastic processes.

\section{Malfunctions}

The simulation also has the ability to accept
malfunction\index{malfunctions} requests (see Section 7.5).  These
requests will change the operating regime of the simulation.  For
example, by causing a crew member to be sick, power supplies to drop,
water to leak, plants to die, etc.  Different modules will have
different sensitivities, for example once plants die they cannot
recover, but if the water module is damaged and then repaired it will
operate normally.  Of course, if levels of CO2, water or food reach
hard-coded critical levels the crew will abandon the mission and return
home.

\section{Crew activity scheduling}
\label{sec:activities}
Crew members in an advanced life support system do not just consume
and produce -- they do activities.  These may be science, maintenance,
exercise, sleep, etc.  The different activities go for a specified length
and for a specified intensity.  The intensity is tied to how many resources the crew
member will consume performing the activity (e.g., sleep takes less 02 to perform than exercise).  
The following diagram shows the nominal daily routine for a crew member:

Schedule loop:
	Sleep for 8 hours
	Hygiene for 1 hour
	Exercise for 1 hour
	Eating for 1 hour
	Mission tasks for 8 hours
	Health for 1 hour
	Maintenance for 1 hour
	Leisure for 2 hours

This schedule can be interrupted by malfunctions (a crew member's activity changed to repair),
sickness, or even mission end (if the crew member hasn't recieved proper resources).

\section{Mass balance}
Mass balanced means resources consumed by components are returned in totality (but usually in a different form) to stores, no mass is lost.
This is important for closed loop simulations as it verfies the models (under perfect conditions) are keeping with the conversation of energy and mass.
Our simulation has been mass balanced to a reasonable degree.  There is a slight loss in mass from the crew (they don't gain weight from eating or produce heat)
and from the plants (who don't produce heat).  We plan to address these issues in the future, but the simulation is currently mass balanced
enough to provide a very decent closed loop simulation.


\chapter{Simulation implementation}

All of the simulation components are written in Java\index{Java} to
make the simulation portable.  Each component can be run as it's own
CORBA\index{CORBA} server so that they can be distributed across
computers for increased performance, or the entire simulation can be run on one server.  We have tested the simulation on
Windows\index{Windows}, Unix\index{Unix} and
Macintosh\index{Macintosh} computers.  A user interface allows for
insight into the simulation which provides real time graphing and more.  There is also a logging\index{logging}
facility that will automatically save all simulation values to a file (xml)
or a database (in development).  We have written Java and C++\index{C++} interfaces to the simulation, though any
language that supports CORBA should be able to access the simulation.

(BioSim) <-> Server ORB <-> Client ORB <-> (Client)

I think we should provide a link to a CORBA tutorial, but assume knowledge of basics of CORBA.
Explain object oriented methodology and how it relates to BioSim.
Would a figure be good here showing the relationship between servers,
clients, CORBA, etc.?

\chapter{Installing the simulation}
The simulation is written in Java and can run on most popular
operating systems.  We provide both an executable for Windows and
source code for any operating system.  This chapter deals with
installing the code onto your computer. 

\section{Installing an executable}

An executable that will run under most Windows operating systems is
available.  You can download the Windows executable from: {\tt
  www.traclabs.com/biosim}.

At that site, go to Download and then download the Windows
installer\index{installation}.  Remember where you put it.
Double-click on the Setup.exe icon, accept the agreement, choose the
default installation, install it where you desire and then click
``Finish''.  BIOSim will start automatically.  NOTE: If you do not
have Java\index{Java} installed on your PC then the installer will
direct you to the WWW page to download it (free).  Please e-mail Scott
Bell (scott@traclabs.com) if you have trouble with installing the
simulation.

\begin{figure}
\insertfigure{labelled_interface}{5.5in}
\caption{BioSim�s graphical user interface.}
\label{fig:empty_interface}
\end{figure}

\section{Installing source code}

Please read the README.txt file in the doc directory of your BIOSIM distribution.

\chapter{Running the simulation}
\label{sec:running}

Doubleclicking on the ``BioSim'' icon in Windows will start the
simulation.  In Linux, run the {\tt run-biosim.sh} shell script in the
{\tt /bin} directory.  When BioSim starts you'll see a blank
interface\index{interface} as shown in
Figure~\ref{fig:empty_interface}.  Holding your mouse over the buttons
on the top reveals what they do.  As noted in the figure, the first
three buttons control the simulation.  The next series of buttons will
display various subsystem views. Some of the views, like the crew, do
not contain anything until the simulation is started.  I like to open
up the main subsystem views before starting the simulation.

The open interface is shown in Figure~\ref{fig:full_interface}.  The
view can be changed between text, chart and schematic by clicking on
the tabs.  The simulation will run until resources (air, water, food,
etc.) drop below critical levels.  Then the simulation will stop.
Pressing the start button will restart the simulation from the
beginning.  If you want to examine the interfaces closely press the
``Pause'' button or go through the simulation one step at a time.

\begin{figure}
\insertfigure{open_interface}{5.5in}
\caption{A full view of BioSim�s graphical user interface}
\label{fig:full_interface}
\end{figure}

\chapter{Configuring the simulation}
The previous chapter discussed installing and running the default
BioSim.  This section will discuss configuring BioSim to address
specific user interests.  This includes changing the underlying
producer/consumer relationship, changing the stochastic nature of
BioSim and changing the logging of data.  We start with some basic
CORBA infrastructure requirements.     

\section{CORBA, Orbs and namesevers}

BioSim is a server.  Client programs connect to BioSim via
CORBA\index{CORBA} -- the Common Object Request Broker Architecture.
In order to connect to the BioSim server will you will need to install
an Object Request Broker (ORB)\index{ORB} on your machine.  We
recommend the JacORB for Java (http://www.jacorb.org) and the ACEOrb for C++ (http://www.cs.wustl.edu/~schmidt/TAO.html), 
although any CORBA-compliant ORB will work.  When the BioSim
server is started it registers with a nameserver\index{nameserver},
which tells client programs where to find it.  The ORBs we recommend
include a name server.  This manual will not cover the details of
CORBA, we refer you to any book on CORBA for your language or look at
the sample programs in Appendix A and B.  We also assume that you have
already installed one of the ORBs.
   
\section{Initial conditions}
\label{sec:init}
BioSim is almost infinitely reconfigurable\index{recongifuration} --
from the number of crew members to the size and number of the
different modules and should be able to provide scenarios ranging from quick shuttle launces to Mars colonies.  
The start-up configuration of BioSim is controlled
by an eXtensible Markup Language (XML)\index{XML} file that is read in
during BioSim initialization.  If you want to change BioSim's
configuration you need to change the XML file.  The default XML file
is located in:

\begin{verbatim}
   BIOSIM/resources/biosim/server/framework/DefaultInitialization.xml
\end{verbatim}

\noindent where {\tt BIOSIM} is the directory where you installed
BioSim.

You can copy this file, change it's name, modify it and then give its
full path name as a parameter to BioSim on startup.  The flag when you
start BioSim is {\tt -xml} followed by the full path name of the XML
file.  If you do not wish to change the initial configuration of
BioSim you can skip the rest of this section. 

\subsection{XML file format}
A snippet from the default initialization XML file is shown below:

\begin{verbatim}
  <SimBioModules>
    <crew>
      <CrewGroup name="CrewGroup">
        <potableWaterConsumer maxFlowRates="100" desiredFlowRates="100" inputs="PotableWaterStore"/>
        <airConsumer maxFlowRates="0" desiredFlowRates="0" inputs="CrewEnvironment"/>
        <foodConsumer maxFlowRates="100'' desiredFlowRates="100" inputs="FoodStore"/>
        <dirtyWaterProducer maxFlowRates="100" desiredFlowRates="100" outputs="DirtyWaterStore"/>
        <greyWaterProducer maxFlowRates="100" desiredFlowRates="100" outputs="GreyWaterStore"/>
        <airProducer maxFlowRates="0" desiredFlowRates="0" outputs="CrewEnvironment"/>
         .
         .
         .
        <crewPerson name="Bob Roberts" age="43" weight="77" sex="MALE">
          <schedule>
            <activity name="sleep" length="8" intensity="1"/>
            <activity name="hygiene" length="1" intensity="2"/>
            <activity name="exercise" length="1" intensity="5"/>
            <activity name="eating" length="1" intensity="2"/>
            <activity name="mission" length="9" intensity="3"/>
            <activity name="health" length="1" intensity="2"/>
            <activity name="maintenance" length="1" intensity="2"/>
            <activity name="leisure" length="2" intensity="2"/>
          </schedule>
        </crewPerson>                               
         .
         .
         .
      </crew>
\end{verbatim}

This file implements the producer/consumer relationship at the heart
of the simulation.  Each module (only the crew module is shown above)
has a name and a list of resources that it consumes and produces along with max
and desired flow rates and what source(s) produce(s) that resource
(the {\tt inputs}) or from where the resource is drawn ({\tt outputs})
More than one source or sink is allowed, but the module tries to draw as much as it can from the first.
In the above snippet some flow rates are 0 because the user has no control over
them (e.g., crew air consumption is based on crew physiology and can't
be directly controlled).  As shown above, you can have as many (or as
few) crew members as you want of various ages, weights and sexes.  You
can give a default schedule for their day (which can be changed via a
CORBA API call during simulation runs). 

The rest of the XML file specification proceeds in a similar way.  The
best way to change the configuration is to start with the default file
and edit it to reflect your requirements.  The default configuration
is based on BIO-Plex\index{BIO-Plex} \cite{tri99}, a NASA JSC ground
testbed for advanced life support.  As such, if all you want is a
``typical'' advanced life support configuration, then just use the
default XML file.

\subsection{Initializing BioSim}

Talk about alternative initialization\index{initialization} via an
API.  Again, capacities, levels, logging, etc. can be set from here.

\section{Enabling stochastic processes}
\index{stochastic}
BioSim provides the ability to change the stochastic nature of the
underlying modules via API calls or the XML initialization file.  If you want a purely deterministic
simulation you can do that.  If you want a very unpredictable
simulation you can do that as well.  This section provides an overview
of the stochastic controls available in BioSim.
The type {\tt StochasticIntensity} includes the following values:

\begin{itemize}
\item {\tt HIGH\_STOCH}
\item {\tt MEDIUM\_STOCH}
\item {\tt LOW\_STOCH}
\item {\tt NONE\_STOCH}
\end{itemize}

Each of these controls the width of the Gaussian\index{Gaussian}
function that provides the variations in output.  Setting the
stochastic intensity to NONE\_STOCH means that the simulation is
deterministic.  All noise is Gaussian.  The outputs of a module as
determined by the module's underlying equations are run through a
Gaussian filter before being output by the simulation.  The Gaussian
filter's parameters are controlled by the stochastic intensity, namely 
the higher the stochastic intensity, the higher the deviation.  

\subsection{Setting logging via the XML configuration file}
To change the stochastic intensity of a module in XML, simply add
setStochasticIntensity as an attribute to the module (if one isn't listed, it defaults to NONE\_STOCH).
example: <AirRS name="AirRS" setStochasticIntensity="HIGH\_STOCH">

The stochastic level for the entire simulation (which is equivalent to
setting each module's stochastic intensity) can by adding setStochasticIntensity to the Globals section at the head
of the XML document.
example:
<Globals crewsToWatch="CrewGroup" setStochasticIntensity="LOW\_STOCH">

\subsection{Setting logging via the API}
Each module of BioSim (i.e., the {\tt SimBioModule} class) has the
following methods:

\begin{itemize}
\item {\tt getStochasticIntensity()}: returns type {\tt
    StochasticIntensity}, which is the current stochastic level of the module.
\item {\tt setStochasticIntensity(StochasticIntensity pIntensity)}:
  Sets the current stochastic intensity of the module to {\tt
  pIntensity}.
\end{itemize}
The stochastic level for the entire simulation (which is equivalent to
setting each module's stochastic intensity) can be set via the {\tt
BioDriver} class using the {\tt setStochasticIntensity} method with
the same parameter ({\tt (StochasticIntensity})

\section{Logging}
\index{logging}

\subsection{Setting logging via the XML configuration file}

To change the logging of a module in XML, add
isLoggingEnabled as an attribute to the module (if one isn't listed, it defaults to false).
example: <AirRS name="AirRS" isLoggingEnabled="true">
You can also globally enable logging for the senors, actuators, or everything by adding
isFullLoggingEnabled, isSensorLoggingEnabled or isActuatorLoggingEnabled to the Globals section at the head
of the XML document.
example:
<Globals crewsToWatch="CrewGroup" isFullLoggingEnabled="true">

\subsection{Setting logging via the API}

Each module of BioSim (i.e., the {\tt SimBioModule} class) has the
following methods:

\begin{itemize}
\item {\tt isLogging()}: returns type {\tt
    boolean}, whether the module is logging.
\item {\tt setLogging(boolean pLogging)}:
  Sets whether the module should log.
\end{itemize}

\begin{figure}
\insertfigure{controller}{4.0in}
\caption{Interaction with the simulation.}
\label{fig:controller}
\end{figure}

\chapter{Controlling the simulation}
The previous two chapters showed you how to run BioSim, monitor its
internal state via a graphical user interface and configure it to your
specifications.  While interesting, this is not useful for control
system research.  This chapter will show you how to connect an
external control program to BioSim and change its underlying
operation.  Note that this chapter will not present an exhaustive list
of all of the objects and methods in BioSim, but only the most
important for new users.  Complete documentation of all objects and
method calls is available at:
http://www.traclabs.com/biosim/doc/api/index.html

Figure~\ref{fig:controller} shows the basic control strategy for the
simulation.  The simulation models represent the underlying equations
that govern the producer/consumer relationships (see
Section~\ref{sec:pc}).  Sensors and actuators are provided to access
the underlying simulations.  The user is responsible for writing the
controller, which uses the sensors and actuators to control the
simulation.  The user can also inject malfunctions into the underlying
simulation models and resupply consumables.   

\section{Controlling simulation runs}
The root class of the simulation is the {\tt BioDriver} class, which
contains the methods used to control the simulation.  The basic
control unit is a {\bf tick}\index{tick}, which advances the
simulation exactly one hour.  After creating your own BioDriver the
following methods are available:

\begin{enumerate}
\item  {\tt void advanceOneTick()}: Moves all modules of the
  simulation ahead exactly one hour.
\item {\tt void endSimulation()}: Ends\index{end} the simulation entirely.
\item {\tt boolean isDone()}: \index{isDone} Returns true if the simulation has met
  an end condition and has stopped.
\item {\tt void reset() }:\index{reset} Returns the simulation to its initial
  state.
\end{enumerate}

Here's an example of using one of these methods after a BioDriver
class has been created (called {\tt myBioDriver}) in a Java program:

\begin{verbatim}

    myBioDriver.advanceOneTick();

\end{verbatim}

There are a number of other methods available in the BioDriver class
that are less commonly used than those listed above.  These are
available via the JavaDoc\index{JavaDoc} facility.  After you have
installed your source code you can make documentation via the {\tt
  make-docs.sh} (for Linux) or {\tt make-docs} (for Windows)
executables in the {\tt bin} directory.  The BioDriverImpl class is
under the biosim.server.framework package.

\section{Accessing sensors and actuators}

\subsection{Generic sensors}

Sensors\index{sensors} are objects that connect to the underlying
simulation.  Each module has its own set of sensor objects, but they
all inherit the same method calls.  These method calls are:

\begin{itemize}

        \item {\tt getMax:} Returns the maximum allowed value of the sensor
        \item {\tt getMin:} Returns the minimum allowed value of the sensor
        \item {\tt getValue}: Returns the current setting of the sensor
\end{itemize}


\subsubsection{Additional sensor methods}

Several other generic methods related to sensor noise, malfunctions
and logging are available:

In addition, there will be some sensor-specific methods, which we will discuss with the appropriate sensor.

\subsection{Generic actuators}

Actuators\index{actuators} are objects that connect to the underlying simulation.  Each
module has its own set of actuator objects, but they all inherit the
same method calls.  These method calls are:

\begin{itemize}

        \item {\tt getMax}: Returns the maximum allowed value of the actuator
        \item {\tt getMin}: Returns the minimum allowed value of the actuator
        \item {\tt getValue}: Returns the current value of the actuator       
        \item {\tt setValue}: Sets the actuator

\end{itemize}

\subsubsection{Additional actuator methods}

Several other generic methods related to actuator noise, malfunctions
and logging are available:

In addition, there will be some actuator-specific methods, which we
will discuss with the appropriate actuator.


\subsection{Environments}

Environments\index{environment} contain the air\index{air} that crew
and crops use.  As there can be multiple environments in the simulation,
there can be multiple sensors tied to each environment.

\subsubsection{Sensors}
\label{sec:env_sen}
The following sensors are available in the environment:

\begin{enumerate}

\item {\tt AirInFlowRateSensor}: Flow rate in moles/tick of air into the environment
\item {\tt AirOutFlowRateSensor}: Flow rate in moles/tick of air out of the environment
\item {\tt CO2AirConcentrationSensor}: Percentage of CO2 in the environment.
\item {\tt CO2AirStoreInFlowRateSensor}: Flow rate in liters/tick of CO2 from a store. (paired with CO2AirEnvironmentInFlowRateSensor)
\item {\tt CO2AirEnvironmentInFlowRateSensor}: Flow rate in liters/tick of CO2 into the environment (paired with CO2AirStoreInFlowRateSensor)
\item {\tt CO2AirPressureSensor}: Pressure in kPA of CO2 in the environment
\item {\tt CO2AirStoreOutFlowRateSensor}: Flow rate in liters/tick of CO2 from a store. (paired with CO2AirEnvironmentOutFlowRateSensor)
\item {\tt CO2AirEnvironmentOutFlowRateSensor}: Flow rate in liters/tick of CO2 out of the environment (paired with CO2AirStoreOutFlowRateSensor)
\item {\tt NitrogenAirConcentrationSensor}: Percentage of Nitrogen in the environment
\item {\tt NitrogenAirEnvironmentInFlowRateSensor}: Flow rate in liters/tick of nitrogen into the environment (paired with NitrogenAirStoreInFlowRateSensor)
\item {\tt NitrogenAirStoreInFlowRateSensor}: Flow rate in liters/tick of nitrogen into a store (paired with NitrogenAirEnvironmentInFlowRateSensor)
\item {\tt NitrogenAirEnvironmentOutFlowRateSensor}: Flow rate in liters/tick of nitrogen out of an environment (paired with NitrogenAirStoreOutFlowRateSensor)
\item {\tt NitrogenAirStoreOutFlowRateSensor}: Flow rate in liters/tick of nitrogen out of a store (paired with NitrogenAirEnvironmentOutFlowRateSensor)
\item {\tt NitrogenAirPressureSensor}: Pressure in kPA of nitrogen in the environment
\item {\tt O2AirConcentrationSensor}: Percentage of O2 in the environment.
\item {\tt O2AirEnvironmentInFlowRateSensor}: Flow rate in liters/tick of O2 into the environment (paired with O2AirStoreInFlowRateSensor)
\item {\tt O2AirStoreInFlowRateSensor}: Flow rate in liters/tick of O2 out of the store (paired with O2AirEnvironmentInFlowRateSensor)
\item {\tt O2AirEnvironmentOutFlowRateSensor}: Flow rate in liters/tick of O2 out of the environment (paired with O2AirStoreOutFlowRateSensor)
\item {\tt O2AirStoreOutFlowRateSensor}: Flow rate in liters/tick of O2 out of the store (paired with O2AirEnvironmentOutFlowRateSensor)
\item {\tt O2AirPressureSensor}: Pressure in kPA of O2 in the environment
\item {\tt OtherAirConcentrationSensor}: Percentage of trace gases in the environment.
\item {\tt OtherAirPressureSensor}: Pressure in kPA of trace gases in the environment
\item {\tt WaterAirConcentrationSensor}: Percentage of water (humidity) in the environment
\item {\tt WaterAirEnvironmentInFlowRateSensor}: Flow rate in liters/tick of water vapor into the environment (paired with WaterAirStoreInFlowRateSensor)
\item {\tt WaterAirStoreInFlowRateSensor}: Flow rate in liters/tick of water vapor into the store (paired with WaterAirEnvironmentInFlowRateSensor)
\item {\tt WaterAirEnvironmentOutFlowRateSensor}: Flow rate in liters/tick of water vapor out of the environment (paired with WaterAirStoreOutFlowRateSensor)
\item {\tt WaterAirStoreOutFlowRateSensor}: Flow rate in liters/tick of water vapor out of the store (paired with WaterAirEnvironmentOutFlowRateSensor)
\item {\tt WaterAirPressureSensor}: Pressure in kPA of water in the environment
\end{enumerate}

\subsubsection{Actuators}

\begin{enumerate}

\item {\tt AirInFlowRateActuator}: Flow rate in moles/tick of air into the environment
\item {\tt AirOutFlowRateActuator}: Flow rate in moles/tick of air out of the environment
\item {\tt CO2AirStoreInFlowRateActuator}: Flow rate in liters/tick of CO2 from a store. (paired with CO2AirEnvironmentInFlowRateActuator)
\item {\tt CO2AirEnvironmentInFlowRateActuator}: Flow rate in liters/tick of CO2 into the environment (paired with CO2AirStoreInFlowRateActuator)
\item {\tt CO2AirStoreOutFlowRateActuator}: Flow rate in liters/tick of CO2 from a store. (paired with CO2AirEnvironmentOutFlowRateActuator)
\item {\tt CO2AirEnvironmentOutFlowRateActuator}: Flow rate in liters/tick of CO2 out of the environment (paired with CO2AirStoreOutFlowRateActuator)
\item {\tt NitrogenAirEnvironmentInFlowRateActuator}: Flow rate in liters/tick of nitrogen into the environment (paired with NitrogenAirStoreInFlowRateActuator)
\item {\tt NitrogenAirStoreInFlowRateActuator}: Flow rate in liters/tick of nitrogen into a store (paired with NitrogenAirEnvironmentInFlowRateActuator)
\item {\tt NitrogenAirEnvironmentOutFlowRateActuator}: Flow rate in liters/tick of nitrogen out of an environment (paired with NitrogenAirStoreOutFlowRateActuator)
\item {\tt NitrogenAirStoreOutFlowRateActuator}: Flow rate in liters/tick of nitrogen out of a store (paired with NitrogenAirEnvironmentOutFlowRateActuator)
\item {\tt O2AirEnvironmentInFlowRateActuator}: Flow rate in liters/tick of O2 into the environment (paired with O2AirStoreInFlowRateActuator)
\item {\tt O2AirStoreInFlowRateActuator}: Flow rate in liters/tick of O2 out of the store (paired with O2AirEnvironmentInFlowRateActuator)
\item {\tt O2AirEnvironmentOutFlowRateActuator}: Flow rate in liters/tick of O2 out of the environment (paired with O2AirStoreOutFlowRateActuator)
\item {\tt O2AirStoreOutFlowRateActuator}: Flow rate in liters/tick of O2 out of the store (paired with O2AirEnvironmentOutFlowRateActuator)
\item {\tt WaterAirEnvironmentInFlowRateActuator}: Flow rate in liters/tick of water vapor into the environment (paired with WaterAirStoreInFlowRateActuator)
\item {\tt WaterAirStoreInFlowRateActuator}: Flow rate in liters/tick of water vapor into the store (paired with WaterAirEnvironmentInFlowRateActuator)
\item {\tt WaterAirEnvironmentOutFlowRateActuator}: Flow rate in liters/tick of water vapor out of the environment (paired with WaterAirStoreOutFlowRateActuator)
\item {\tt WaterAirStoreOutFlowRateActuator}: Flow rate in liters/tick of water vapor out of the store (paired with WaterAirEnvironmentOutFlowRateActuator)

\end{enumerate}

\subsubsection{Malfunctions}
State exactly what the malfunctions do.  How do the method calls work.
What is the malfunction. What is the leak rate, etc. for each of the
different malfunction classes.  Since method calls are probably
identical may want to move this up to top and then use this section
only for module-specific malfunction information.

\subsection{Crew}
\index{crew}
\subsubsection{Sensors}

\begin{enumerate}

\item {\tt CrewGroupActivitySensor}:  Returns average activity level of crew
\item {\tt CrewGroupAnyDeadSensor}: Returns whether any crew member has died
\item {\tt CrewGroupDeathSensor}: Return whether the entire crew has died
\item {\tt CrewGroupProductivitySensor}: Total productivity of the crew

\end{enumerate}

\subsubsection{Actuators}

\begin{enumerate}

\item {\tt CrewGroupActivityActuator}:

\end{enumerate}

\subsubsection{Malfunctions}

\subsection{Air}
\index{air}
\subsubsection{Sensors}

\begin{enumerate}
\item {\tt CO2InFlowRateSensor}: Reports the CO2 flow rate (in moles) into a module
\item {\tt CO2OutFlowRateSensor}: Reports the CO2 flow rate (in moles) out of a module
\item {\tt CO2StoreLevelSensor}: How much CO2 (in moles) is contained in this store.
\item {\tt H2InFlowRateSensor}: Reports the H2 flow rate (in moles) into a module
\item {\tt H2OutFlowRateSensor}: Reports the H2 flow rate (in moles) out of a module
\item {\tt H2StoreLevelSensor}: How much   (in moles) is contained in this store.
\item {\tt NitrogenInFlowRateSensor}: Reports the nitrogen flow rate (in moles) into a module
\item {\tt NitrogenOutFlowRateSensor}: Reports the nitrogen flow rate (in moles) out of a module
\item {\tt NitrogenStoreLevelSensor}: How much nitrogen (in moles) is contained in this store.
\item {\tt O2InFlowRateSensor}: Reports the O2 flow rate (in moles) into a module
\item {\tt O2OutFlowRateSensor}: Reports the O2 flow rate (in moles) out of a module
\item {\tt O2StoreLevelSensor}: How much O2 (in moles) is contained in this store.
\end{enumerate}

\subsubsection{Actuators}

\begin{enumerate}
\item {\tt CO2InFlowRateAcutator}: Changes how much CO2 (in moles) goes into the module
\item {\tt CO2OutFlowRateAcutator}: Changes how much CO2 (in moles) goes out of the module
\item {\tt H2InFlowRateAcutator}: Changes how much H2 (in moles) goes into the module
\item {\tt H2OutFlowRateAcutator}: Changes how much H2 (in moles) goes out of the module
\item {\tt NitrogenInFlowRateAcutator}: Changes how much nitrogen (in moles) goes into the module
\item {\tt NitrogenOutFlowRateAcutator}: Changes how much nitrogen (in moles) goes out of the module
\item {\tt O2InFlowRateAcutator}: Changes how much O2 (in moles) goes into the module
\item {\tt O2OutFlowRateAcutator}: Changes how much O2 (in moles) goes out of the module
\end{enumerate}

\subsubsection{Malfunctions}

Same as others


\subsection{Water}
\index{water}

\subsubsection{Sensors}
\begin{enumerate}
\item {\tt DirtyWaterInFlowRateSensor}: Reports the dirty water flow rate (in liters) into a module
\item {\tt DirtyWaterOutFlowRateSensor}: Reports the dirty water flow rate (in liters) out of a module
\item {\tt DirtyWaterStoreLevelSensor}: How much dirty water (in liters) is contained in this store.
\item {\tt GreyWaterInFlowRateSensor}: Reports the grey water flow rate (in liters) into a module
\item {\tt GreyWaterOutFlowRateSensor}: Reports the grey water flow rate (in liters) out of a module
\item {\tt GreyWaterStoreLevelSensor}: How much grey water (in liters) is contained in this store.
\item {\tt PotableWaterInFlowRateSensor}: Reports the potable water flow rate (in liters) into a module
\item {\tt PotableWaterOutFlowRateSensor}: Reports the potable water flow rate (in liters) out of a module
\item {\tt PotableWaterStoreLevelSensor}: How much potable water (in liters) is contained in this store.
\end{enumerate}

\subsubsection{Actuators}

\begin{enumerate}
\item {\tt DirtyWaterInFlowRateActuator}: Changes how much dirty water (in liters) goes into the module
\item {\tt DirtyWaterOutFlowRateActuator}: Changes how much dirty water (in liters) goes ouf of the module
\item {\tt GreyWaterInFlowRateActuator}: Changes how much grey water (in liters) goes into the module
\item {\tt GreyWaterOutFlowRateActuator}: Changes how much grey water (in liters) goes ouf of the module
\item {\tt PotableWaterInFlowRateActuator}: Changes how much potable water (in liters) goes into the module
\item {\tt PotableWaterOutFlowRateActuator}: Changes how much potable water (in liters) goes ouf of the module
\end{enumerate}

\subsubsection{Malfunctions}

\subsection{Biomass}
\index{biomass}

\subsubsection{Sensors}
\begin{enumerate}
\item {\tt BiomassInFlowRateSensor}: Reports the biomass flow rate (in kilograms) into a module
\item {\tt BiomassOutFlowRateSensor}: Reports the biomass flow rate (in kilograms) out of a module
\item {\tt BiomassStoreLevelSensor}: How much biomass (in kilograms) is contained in this store.
\item {\tt FoodInFlowRateSensor}: Reports the food flow rate (in kilograms) into a module
\item {\tt FoodOutFlowRateSensor}: Reports the food flow rate (in kilograms) out of a module
\item {\tt FoodStoreLevelSensor}: How much food (in kilograms) is contained in this store.
\end{enumerate}

\subsubsection{Actuators}
\begin{enumerate}
\item {\tt BiomassInFlowRateActuator}: Changes how much biomass (in kilograms) goes into the module
\item {\tt BiomassOutFlowRateActuator}: Changes how much biomass (in kilograms) goes out of the module
\item {\tt FoodInFlowRateActuator}: Changes how much food (in kilograms) goes into the module
\item {\tt FoodOutFlowRateActuator}: Changes how much food (in kilograms) goes out of the module
\end{enumerate}

\subsubsection{Malfunctions}

\subsection{Dry Waste}
\index{waste}
\subsubsection{Sensors}
\begin{enumerate}
\item {\tt DryWasteInFlowRateSensor}: Changes how much dry Waste (in kilograms) goes into the module
\item {\tt DryWasteOutFlowRateSensor}: Changes how much dry Waste (in kilograms) goes out of the module
\item {\tt DryWasteStoreLevelSensor}: How much dry Waste (in kilograms) is contained in this store.
\end{enumerate}

\subsubsection{Actuators}
\begin{enumerate}
\item {\tt DryWasteInFlowRateActuator}: Changes how much dry waste (in kilograms) goes into the module
\item {\tt DryWasteOutFlowRateActuator}: Changes how much dry waste (in kilograms) goes out of the the module
\end{enumerate}

\subsubsection{Malfunctions}

\subsection{Power}
\index{power}

\subsubsection{Sensors}

\begin{enumerate}
\item {\tt PowerInFlowRateSensor}: Changes how much power (in watts) goes into the module
\item {\tt PowerOutFlowRateSensor}: Changes how much power (in watts) goes out of the module
\item {\tt PowerStoreLevelSensor}: How much power (in watts) is contained in this store.
\end{enumerate}

\subsubsection{Actuators}

\begin{enumerate}
\item {\tt PowerInFlowRateActuator}: Changes how much power (in watts) goes into the module
\item {\tt PowerOutFlowRateActuator}: Changes how much power (in watts) goes out of the module
\end{enumerate}

\subsubsection{Malfunctions}



\chapter{Writing a simple Java controller}

A short tutorial on creating a BioSim, initializing it and then
calling some sensors, actuators and ticks.  Assume JacOrb. 
See BIOSIM/doc/TestBiosim.java

\chapter{Writing a simple C++ controller}



Same as above in C++.  Assume ACEOrb.  Assume Linux (?).

\chapter{Control Examples}

Include GA, Rice learners, Texas Tech learner.

\chapter{Conclusions}

\appendix 
\chapter{Java Example}

\begin{verbatim}

Put parts of Rice handcontroller here

\end{verbatim}

\chapter{C++ Example}

\begin{verbatim}

#include "biosimC.h"
#include <CosNamingC.h>

static const char* CorbaServerName = "BioDriver0";
int
main(int argc,char *argv[])
{
  try {
    // Initialize orb
    CORBA::ORB_var orb = CORBA::ORB_init(argc,argv,"TestClient");

    cerr << "initialized orb" << endl;

    if (argc < 2) {
      cerr << "Usage: " << argv[0]
           << " IOR_string" << endl;
      return 1;
    }

// Use the first argument to create the object reference,
    // in real applications we use the naming service, but let's do
    // the easy part first!

    CORBA::Object_var obj =
      orb->string_to_object (argv[1]); 

    cout << "After resolve_initial_references" << endl;


    /*   CORBA::Object_var obj =
         orb->string_to_object (argv[1]); */
 
    CosNaming::NamingContext_var inc = CosNaming::NamingContext::_narrow(obj);

    cout << "After NameService narrow" << endl;

    if(CORBA::is_nil(inc)) {
      cerr << "Unable to narrow NameService." << endl;
      throw 0;
    }  // end if couldn't get root naming context
    
    CosNaming::Name name;
    name.length(1);
    name[0].id = CORBA::string_dup(CorbaServerName);
    name[0].kind = CORBA::string_dup("");

    CORBA::Object_var tmp = inc->resolve(name);

    cout << "After resolve name" << endl;

    if(CORBA::is_nil(tmp)) {
      cerr << "Unable to resolve " << CorbaServerName << " from NameService" << endl;
      throw 0;
    }  // end if couldn't find object

    cerr << "got object" << endl;

    // Now downcast the object reference to the appropriate type

    biosim::idl::framework::BioDriver_var biodriver = biosim::idl::framework::BioDriver::_narrow(tmp);


    cerr << "narrowed" << endl;

    // code here

    biodriver->spawnSimulationTillDead();

    cerr << "spawned" << endl;

    orb->destroy ();
  }
  catch (CORBA::Exception &ex) {
    std::cerr << "CORBA exception raised! " << std::endl;
  }
  return 0;
}

\end{verbatim}

\bibliography{biosim}
\bibliographystyle{plain}

\begin{theindex}

\documentstyle{report}

\input epsf

%\nofiles

\setlength{\textheight}{9.0in}
\setlength{\textwidth}{6.5in}
\setlength{\footheight}{0.0in}
\setlength{\topmargin}{0.00in}
\setlength{\headheight}{0.0in}
\setlength{\headsep}{0.0in}
\setlength{\oddsidemargin}{-0.0in}

\makeindex

\begin{document}

\def\topfraction{1.0}        %maximum fraction of floats at the top of the page
\def\bottomfraction{1.0}     %ditto, for the bottom of the page
\def\textfraction{0}         %minimum fraction of text (--> 100% floats is okay)
\def\floatpagefraction{1.0}  % if a page is full of floats, it'd better be FULL

\newcommand{\insertfigure}[2]{\begin{center}
                             \ \epsfxsize=#2 \epsfbox{#1.ps}
                             \end{center}}

%\newcommand{\insertfigure}[2]{\vspace{#2}}

\title{BioSim: An Integrated Simulation of an Advanced Life Support System for Intelligent Control Research\\
Users Manual (draft)}

\author{Metrica Inc. and S\&K Technologies\\
NASA Johnson Space Center/ER2\\
Houston TX 77058\\
http://www.traclabs.com/biosim}


\date{\today}

\maketitle

\tableofcontents

\newpage

\listoffigures

\newpage

\chapter{Introduction}
Advanced life support systems have multiple interacting subsystems,
which makes control a particularly challenging task.  The simulation
described in this document provides a testbed for integrated control
research.  There have been other integrated life support simulations
(e.g., \cite{finn99}) and we have learned from those efforts.  This
simulation is designed exclusively for integrated controls research,
which imposes different requirements.  For example, the simulation is
accessed through sensors and actuators, just as a real system would
be.  Noise and uncertainty are built in and controllable.
Malfunctions and failures of subsystems are modeled and manifest
themselves through anomalous readings in the sensors.  Crew members
are taskable and their tasks have purpose and meaning in the
simulation.  In essence, the simulation is a replacement for the
Advanced Life Support (ALS) hardware and crew, allowing for testing of
control approaches in advance of any integrated test.  We want the
simulation to be used to develop and evaluate integrated control
techniques.  There are still many open research questions with respect
to controlling advanced life support systems.  For example is a
distributed or hierarchical approach better?  What role does machine
learning play in control?  How can symbolic, qualitative control
approaches be integrated with continuous, quantitative approaches?
How can we evaluate different control philosophies?  We hope that the
controls community can use this simulation to begin to build systems
which will provide us with answers to these kinds of questions. 

The simulation is written in Java and is accessed using the Common
Object Request Broker Architecture (CORBA)\index{CORBA}.  The distributed nature of
the simulation allows for multiple instances to be run in parallel,
making it ideal for testing advanced control concepts such as genetic
algorithms or reinforcement learning, which require multiple trials.
See \cite{kortenkamp_etal01} or \cite{schreckenghost_etal02} for a
background on advanced life support control issues and previous work.
This document begins with a brief introduction to advanced life
support systems and provides details on installing, running and
interacting with the simulation.  The simulation is available from:

\begin{verbatim}
     http://www.traclabs.com/biosim
\end{verbatim}

\chapter{Background on advanced life support}

An advanced life support system is one in which many resources are
reused or regenerated.  Such systems will be necessary for long
duration human missions, such as those to Mars, where resupply
opportunities are limited.  Typically they have thin margins for mass,
power and buffers, which requires optimization and tight control.
Also, advanced life support systems consist of many interconnected
subsystems all of which interact in both predictable and unpredictable
ways (see Figure~\ref{fig:biosim}).  Autonomous control\index{control}
of these systems is desirable. This section gives enough background so
that users of the simulation can understand the major components of an
advanced life support system.  For detailed documentation on advanced
life support systems see \cite{tri99} or go to:

\begin{verbatim}
    http://advlifesupport.jsc.nasa.gov
\end{verbatim}

\begin{figure}
\insertfigure{figure}{5.5in}
\caption{The various modules that comprise the BIOSim integrated simulation.}
\label{fig:biosim}
\end{figure}

\section{Modules}

An advanced life support system consists of many interacting
subsystems.  While each is self-contained, they rely on each other for
various resources.  The simulation is built from a set of modules that
each produce and consume different resources.

\subsection{Environment}
\label{sec:env}
An environment\index{environment} contains air that is consumed by
either people or crops.  Air\index{air} contains a mixture of gases --
in our simulation these gases are oxygen (O2), carbon dioxide (CO2),
nitrogen (N), water vapor (H20), and other gasses (trace).  The initial composition of the
gases is set by the simulation initialization file (see
Section~\ref{sec:init}).  As the simulation runs modules may consume
air from the simulation and replace it with air of a different
composition.  Thus, the composition of gases and pressure in the air change over
time and can be measured by environment sensors
(Section~\ref{sec:env_sen}).  As with all modules, there can be
multiple environments.  For example, it is common for crew members
and crops to have different air compositions and that is the default
in this simulation.   

\subsection{Crew}
\label{sec:crew}

The crew\index{crew} module is implemented using models described in
\cite{goudarzi_ting99}.  The number, gender, age and weight of the
crew are settable as input parameters -- in the default configuration
there are four crew members, two male and two female.  The crew cycles
through a set of activities (sleep, maintenance, recreation, etc.).
As they do so they consume O2, food and water and produce CO2, dirty
water and solid waste.  The amount of resources consumed and produced
varies according to crew member attributes and their activities.  The
crew's activities can be adjusted by passing a new crew schedule to
the crew module.  A default schedule can also be used.  The crew
module is connected to a crew environment that contains an atmosphere
that they breathe.  The initial size and gas composition (percentages
of O2, CO2, H2O and inert gases) are input parameters and the default
is $1.54893 x 10^6 liters$ (from \cite{tri99}) with an atmosphere
equivalent to sea level air.  As the simulation progresses the mixture
of gases in the atmosphere changes.

\subsection{Water}
\label{sec:water}
Consumes:Power, Grey Water, Dirty Water
Produces: Potable Water

The water\index{water} recovery module consumes dirty water, grey water(i.e., water that can be used for
washing but not drinking), and power and
produces potable water.  The water recovery module consists of four
subsystems that process the water.  The biological water processing
(BWP) subsystem removes organic compounds.  Then the water passes to a
reverse osmosis (RO) subsystem, which makes 85\% of the water passing
through it grey.  The 15\% of water remaining from the RO (called
brine) is passed to the air evaporation subsystem (AES), which
recovers the rest.  These two streams of grey water (from the RO and
the AES) are passed through a post-processing subsystem (PPS) to
create potable water.  An external controller can turn on or off
various subsystems.  For example, all water can pass through the AES
at a higher power cost.  We based our water recovery module on a
recently completed test at NASA Johnson Space Center
\cite{bonasso_etal03}.

\subsection{Air}
\label{sec:air}
Consumes:Power, H2, Air, CO2, Potable Water
Produces: Air (with less CO2), O2, H2, CO2, Potable Water

The air\index{air} component takes in exhalant CO2 and produces O2 as long as
there is sufficient power being provided to the system.  This
component is modeled on various Air Revitalization System (ARS) work
at NASA JSC \cite{malin_etal00}.  There are three interacting air
subsystems: the Variable Configuration Carbon Dioxide Removal (VCCR)
System in which CO2 is removed from the air stream; the Carbon Dioxide
Reduction System (CRS), which also removes CO2 from the air stream
using a different process and producing different gases than the VCCR;
and the Oxygen Generation System (OGS) in which O2 is added to the air
stream by breaking water down into hydrogen and oxygen.  It is
important to note that both the removal of CO2 and the addition of O2
are required for human survival.  It is also important to note that
the biomass component (next subsection) also removes CO2 and adds O2.

\subsection{Biomass}
\label{sec:biomass}
Consumes:Power, Potable Water, Grey Water, Air
Produces: Air (with more CO2), Biomass, Dirty Water, CO2, Potable Water

The biomass\index{biomass} component is where crops\index{crops} are
grown.  It produces both biomass, which can be turned into
food\index{food}, and O2 and consumes water, power (light) and CO2.
This component is based on models given in \cite{jones_cavazzoni00}.
The system is modeled as shelves\index{shelves} that contain plants,
lights and water.  Shelves are planted and harvested and there is
growth cycle for each shelf.  Currently, ten crops are modeled and can
be planted in any ratio.

\subsection{Food processing}
\label{sec:food}
Consumes:Power, Biomass
Produces: Food

Before biomass can be consumed by the crew it must be converted to
food\index{food}.  The food processing component takes biomass, power and crew
time and produces food and solid waste.  The crew needs to be involved
in this process as it is labor intensive.  See Section~\ref{sec:crew}
for more information on scheduling crew activities.

\subsection{Waste}
\label{sec:waste}
Consumes:Power, Dry Waste, O2
Produces: CO2

The waste\index{waste} component consumes power, O2 and solid waste
and produces CO2.  It is modeled on an incinerator used in the Phase
III test in 1997 \cite{schreckenghost_etal02}.  Incineration can be
scheduled.

\subsection{Thermal control}
\label{sec:thermal}
Consumes: ?
Produces: ?

The thermal control\index{thermal} component regulates the air temperature in the
habitat.  This component is not implemented for this simulation and it
is assumed that external controllers are maintaining chamber
temperature.

\subsection{Power}
\label{sec:power}
Consumes: 
Produces: Power

The power\index{power} component supplies power to all of the other
components that need it.  There are two choices for power in the
simulation.  The first is nuclear\index{nuclear} power, which supplies
a fixed amount throughout the lifetime of the simulation.  The second
is solar\index{solar} power, which supplies a varying amount
(day/night cycle) of power to each component.

\subsection{Accumulators}
\label{sec:accumulator}
Consumes: All
Produces: All

The accumulator\index{accumulator} component can take a resource from any
store or environment and place it into another environment or store.  It is
functionally equivalent to an injector.

\subsection{Injectors}
\label{sec:injector}
Consumes: All
Produces: All

The injector\index{injector} component can take a resource from any
store or environment and place it into another environment or store.  It is
functionally equivalent to an accumulator.

\chapter{Simulation Properties}

The simulation implements the modules outlined in the previous
section.  BioSim does {\em not} simulate at the level of valves,
pumps, switches, etc.  Instead, modules are implemented in a
producer/consumer relationship, which is described in the next
subsection.  The simulation also provides additional functionality to
help test and debug control programs.

\section{Producer/Consumer Model}

No component in BioSim directly interacts with other components.  Instead, each component
has a rigid set of resources that it consumes and produces.  The resources are taken from stores/environments
and put into stores/environments.  The resources for BioSim currently consist of
power, potable water, grey water, dirty water, air, H2, nitrogen, O2, CO2, biomass, food, and dry waste.  At each simulation tick,
each module takes resource from it's store (it's consumables), and puts resources into stores (it's products).  Stores always report
thier values (level and capacity) from the last tick and report new values after every component in the simulation has been ticked.  
Resource conflicts can occur when two components ask for a limited resource, i.e. two WaterRS's both need 20 liters of dirty water and there
is only 20 liters in the dirty water tank.  This is currently resolved on a winner-take-all basis that is for all practical purposes
randomized.

\section{Success criteria}

In order to compare different control approaches there should be
objective criteria for a successful advanced life support system
mission.  Several possibilities exist already in the simulation.  For
example, the length of the mission before consumables are gone is a
success criterion.  As is minimizing the starting levels of stores or
minimizing the sizes of intermediary buffers.  However, many of these
fail to get at the true success of a mission, the productivity of the
crew in performing their science objectives.  Thus, the simulation
includes an artificial productivity\index{productivity} measure.  As
the crew cycles through their activities (see Section~\ref{sec:activities})
the amount of time they spend doing "mission" tasks is accumulated. This
number is also multiplied by a factor that takes into account the
amount of sleep, exercise, water, food and oxygen they are getting to approximate 
crew effectivness at performing the task (a happy crew is a productive crew!).
So each tick, for each crew member the productivity is given by the
following equation:

        [need to put the equation here]
	[some relevant code, derive equation later]
	\begin{verbatim}
	private static final float WATER_TILL_DEAD = 8.1f;
	private static final float WATER_RECOVERY_RATE=0.01f;
	private static final float OXYGEN_TILL_DEAD = 3f;
	private static final float OXYGEN_RECOVERY_RATE=0.01f;
	private static final float CALORIE_TILL_DEAD = 180000f;
	private static final float CALORIE_RECOVERY_RATE=0.0001f;
	private static final float DANGEROUS_CO2_RATION = 0.06f;
	private static final float CO2_TILL_DEAD = 10f;
	private static final float CO2_RECOVERY_RATE=0.001f;
	private static final float LEISURE_TILL_BURNOUT = 168f;
	private static final float LEISURE_RECOVERY_RATE=90f;
	private static final float AWAKE_TILL_EXHAUSTION = 120f;
	private static final float SLEEP_RECOVERY_RATE=12f;
	consumedWaterBuffer = new SimpleBuffer(WATER_TILL_DEAD, WATER_TILL_DEAD);
	consumedOxygenBuffer = new SimpleBuffer(OXYGEN_TILL_DEAD, OXYGEN_TILL_DEAD);
	consumedCaloriesBuffer = new SimpleBuffer(CALORIE_TILL_DEAD, CALORIE_TILL_DEAD);
	consumedCO2Buffer = new SimpleBuffer(CO2_TILL_DEAD, CO2_TILL_DEAD);
	sleepBuffer = new SimpleBuffer(AWAKE_TILL_EXHAUSTION, AWAKE_TILL_EXHAUSTION);
	leisureBuffer = new SimpleBuffer(LEISURE_TILL_BURNOUT, LEISURE_TILL_BURNOUT);
	float caloriePercentFull = sigmoidLikeProbability(consumedCaloriesBuffer.getLevel() / consumedCaloriesBuffer.getCapacity());
	float waterPercentFull = sigmoidLikeProbability(consumedWaterBuffer.getLevel() / consumedWaterBuffer.getCapacity());
	float oxygenPercentFull = sigmoidLikeProbability(consumedOxygenBuffer.getLevel() / consumedOxygenBuffer.getCapacity());
	float CO2PercentFull = sigmoidLikeProbability(consumedCO2Buffer.getLevel() / consumedCO2Buffer.getCapacity());
	float sleepPercentFull = sleepBuffer.getLevel() / sleepBuffer.getCapacity();
	float leisurePercentFull = leisureBuffer.getLevel() / leisureBuffer.getCapacity();
	float averagePercentFull = (caloriePercentFull + waterPercentFull + oxygenPercentFull + CO2PercentFull + sleepPercentFull + leisurePercentFull) / 6f;
	myMissionProductivity += myCrewGroup.randomFilter(averagePercentFull);
	\end{verbatim}
\section{Stochastic processes}

Any sufficiently complex process, especially one with biological
components, will not be deterministic\index{deterministic}.  That is,
given the same starting conditions and inputs it will not produce the
exact same outputs each time it is run.  A
stochastic\index{stochastic} process is one in which chance or
probability affect the outcome.  This simulation offers both
deterministic and stochastic operations.  In the former case, running
the simulation twice with the same inputs and initial conditions will
produce the same results.  This may be useful for quantitatively
comparing different control approaches against each other.  In the
latter case, the user can control the amount of variability in the
simulation (see Section 7.4).  This is modeled using a Gaussian\index{Gaussian}
distribution.  The deviation is determined by the stochastic intensity; a higher intensity will yield a higher deviation.  The filter is then appropriately applied to certain variable of the model.  This can be used to test a
control system�s ability to deal with stochastic processes.

\section{Malfunctions}

The simulation also has the ability to accept
malfunction\index{malfunctions} requests (see Section 7.5).  These
requests will change the operating regime of the simulation.  For
example, by causing a crew member to be sick, power supplies to drop,
water to leak, plants to die, etc.  Different modules will have
different sensitivities, for example once plants die they cannot
recover, but if the water module is damaged and then repaired it will
operate normally.  Of course, if levels of CO2, water or food reach
hard-coded critical levels the crew will abandon the mission and return
home.

\section{Crew activity scheduling}
\label{sec:activities}
Crew members in an advanced life support system do not just consume
and produce -- they do activities.  These may be science, maintenance,
exercise, sleep, etc.  The different activities go for a specified length
and for a specified intensity.  The intensity is tied to how many resources the crew
member will consume performing the activity (e.g., sleep takes less 02 to perform than exercise).  
The following diagram shows the nominal daily routine for a crew member:

Schedule loop:
	Sleep for 8 hours
	Hygiene for 1 hour
	Exercise for 1 hour
	Eating for 1 hour
	Mission tasks for 8 hours
	Health for 1 hour
	Maintenance for 1 hour
	Leisure for 2 hours

This schedule can be interrupted by malfunctions (a crew member's activity changed to repair),
sickness, or even mission end (if the crew member hasn't recieved proper resources).

\section{Mass balance}
Mass balanced means resources consumed by components are returned in totality (but usually in a different form) to stores, no mass is lost.
This is important for closed loop simulations as it verfies the models (under perfect conditions) are keeping with the conversation of energy and mass.
Our simulation has been mass balanced to a reasonable degree.  There is a slight loss in mass from the crew (they don't gain weight from eating or produce heat)
and from the plants (who don't produce heat).  We plan to address these issues in the future, but the simulation is currently mass balanced
enough to provide a very decent closed loop simulation.


\chapter{Simulation implementation}

All of the simulation components are written in Java\index{Java} to
make the simulation portable.  Each component can be run as it's own
CORBA\index{CORBA} server so that they can be distributed across
computers for increased performance, or the entire simulation can be run on one server.  We have tested the simulation on
Windows\index{Windows}, Unix\index{Unix} and
Macintosh\index{Macintosh} computers.  A user interface allows for
insight into the simulation which provides real time graphing and more.  There is also a logging\index{logging}
facility that will automatically save all simulation values to a file (xml)
or a database (in development).  We have written Java and C++\index{C++} interfaces to the simulation, though any
language that supports CORBA should be able to access the simulation.

(BioSim) <-> Server ORB <-> Client ORB <-> (Client)

I think we should provide a link to a CORBA tutorial, but assume knowledge of basics of CORBA.
Explain object oriented methodology and how it relates to BioSim.
Would a figure be good here showing the relationship between servers,
clients, CORBA, etc.?

\chapter{Installing the simulation}
The simulation is written in Java and can run on most popular
operating systems.  We provide both an executable for Windows and
source code for any operating system.  This chapter deals with
installing the code onto your computer. 

\section{Installing an executable}

An executable that will run under most Windows operating systems is
available.  You can download the Windows executable from: {\tt
  www.traclabs.com/biosim}.

At that site, go to Download and then download the Windows
installer\index{installation}.  Remember where you put it.
Double-click on the Setup.exe icon, accept the agreement, choose the
default installation, install it where you desire and then click
``Finish''.  BIOSim will start automatically.  NOTE: If you do not
have Java\index{Java} installed on your PC then the installer will
direct you to the WWW page to download it (free).  Please e-mail Scott
Bell (scott@traclabs.com) if you have trouble with installing the
simulation.

\begin{figure}
\insertfigure{labelled_interface}{5.5in}
\caption{BioSim�s graphical user interface.}
\label{fig:empty_interface}
\end{figure}

\section{Installing source code}

Please read the README.txt file in the doc directory of your BIOSIM distribution.

\chapter{Running the simulation}
\label{sec:running}

Doubleclicking on the ``BioSim'' icon in Windows will start the
simulation.  In Linux, run the {\tt run-biosim.sh} shell script in the
{\tt /bin} directory.  When BioSim starts you'll see a blank
interface\index{interface} as shown in
Figure~\ref{fig:empty_interface}.  Holding your mouse over the buttons
on the top reveals what they do.  As noted in the figure, the first
three buttons control the simulation.  The next series of buttons will
display various subsystem views. Some of the views, like the crew, do
not contain anything until the simulation is started.  I like to open
up the main subsystem views before starting the simulation.

The open interface is shown in Figure~\ref{fig:full_interface}.  The
view can be changed between text, chart and schematic by clicking on
the tabs.  The simulation will run until resources (air, water, food,
etc.) drop below critical levels.  Then the simulation will stop.
Pressing the start button will restart the simulation from the
beginning.  If you want to examine the interfaces closely press the
``Pause'' button or go through the simulation one step at a time.

\begin{figure}
\insertfigure{open_interface}{5.5in}
\caption{A full view of BioSim�s graphical user interface}
\label{fig:full_interface}
\end{figure}

\chapter{Configuring the simulation}
The previous chapter discussed installing and running the default
BioSim.  This section will discuss configuring BioSim to address
specific user interests.  This includes changing the underlying
producer/consumer relationship, changing the stochastic nature of
BioSim and changing the logging of data.  We start with some basic
CORBA infrastructure requirements.     

\section{CORBA, Orbs and namesevers}

BioSim is a server.  Client programs connect to BioSim via
CORBA\index{CORBA} -- the Common Object Request Broker Architecture.
In order to connect to the BioSim server will you will need to install
an Object Request Broker (ORB)\index{ORB} on your machine.  We
recommend the JacORB for Java (http://www.jacorb.org) and the ACEOrb for C++ (http://www.cs.wustl.edu/~schmidt/TAO.html), 
although any CORBA-compliant ORB will work.  When the BioSim
server is started it registers with a nameserver\index{nameserver},
which tells client programs where to find it.  The ORBs we recommend
include a name server.  This manual will not cover the details of
CORBA, we refer you to any book on CORBA for your language or look at
the sample programs in Appendix A and B.  We also assume that you have
already installed one of the ORBs.
   
\section{Initial conditions}
\label{sec:init}
BioSim is almost infinitely reconfigurable\index{recongifuration} --
from the number of crew members to the size and number of the
different modules and should be able to provide scenarios ranging from quick shuttle launces to Mars colonies.  
The start-up configuration of BioSim is controlled
by an eXtensible Markup Language (XML)\index{XML} file that is read in
during BioSim initialization.  If you want to change BioSim's
configuration you need to change the XML file.  The default XML file
is located in:

\begin{verbatim}
   BIOSIM/resources/biosim/server/framework/DefaultInitialization.xml
\end{verbatim}

\noindent where {\tt BIOSIM} is the directory where you installed
BioSim.

You can copy this file, change it's name, modify it and then give its
full path name as a parameter to BioSim on startup.  The flag when you
start BioSim is {\tt -xml} followed by the full path name of the XML
file.  If you do not wish to change the initial configuration of
BioSim you can skip the rest of this section. 

\subsection{XML file format}
A snippet from the default initialization XML file is shown below:

\begin{verbatim}
  <SimBioModules>
    <crew>
      <CrewGroup name="CrewGroup">
        <potableWaterConsumer maxFlowRates="100" desiredFlowRates="100" inputs="PotableWaterStore"/>
        <airConsumer maxFlowRates="0" desiredFlowRates="0" inputs="CrewEnvironment"/>
        <foodConsumer maxFlowRates="100'' desiredFlowRates="100" inputs="FoodStore"/>
        <dirtyWaterProducer maxFlowRates="100" desiredFlowRates="100" outputs="DirtyWaterStore"/>
        <greyWaterProducer maxFlowRates="100" desiredFlowRates="100" outputs="GreyWaterStore"/>
        <airProducer maxFlowRates="0" desiredFlowRates="0" outputs="CrewEnvironment"/>
         .
         .
         .
        <crewPerson name="Bob Roberts" age="43" weight="77" sex="MALE">
          <schedule>
            <activity name="sleep" length="8" intensity="1"/>
            <activity name="hygiene" length="1" intensity="2"/>
            <activity name="exercise" length="1" intensity="5"/>
            <activity name="eating" length="1" intensity="2"/>
            <activity name="mission" length="9" intensity="3"/>
            <activity name="health" length="1" intensity="2"/>
            <activity name="maintenance" length="1" intensity="2"/>
            <activity name="leisure" length="2" intensity="2"/>
          </schedule>
        </crewPerson>                               
         .
         .
         .
      </crew>
\end{verbatim}

This file implements the producer/consumer relationship at the heart
of the simulation.  Each module (only the crew module is shown above)
has a name and a list of resources that it consumes and produces along with max
and desired flow rates and what source(s) produce(s) that resource
(the {\tt inputs}) or from where the resource is drawn ({\tt outputs})
More than one source or sink is allowed, but the module tries to draw as much as it can from the first.
In the above snippet some flow rates are 0 because the user has no control over
them (e.g., crew air consumption is based on crew physiology and can't
be directly controlled).  As shown above, you can have as many (or as
few) crew members as you want of various ages, weights and sexes.  You
can give a default schedule for their day (which can be changed via a
CORBA API call during simulation runs). 

The rest of the XML file specification proceeds in a similar way.  The
best way to change the configuration is to start with the default file
and edit it to reflect your requirements.  The default configuration
is based on BIO-Plex\index{BIO-Plex} \cite{tri99}, a NASA JSC ground
testbed for advanced life support.  As such, if all you want is a
``typical'' advanced life support configuration, then just use the
default XML file.

\subsection{Initializing BioSim}

Talk about alternative initialization\index{initialization} via an
API.  Again, capacities, levels, logging, etc. can be set from here.

\section{Enabling stochastic processes}
\index{stochastic}
BioSim provides the ability to change the stochastic nature of the
underlying modules via API calls or the XML initialization file.  If you want a purely deterministic
simulation you can do that.  If you want a very unpredictable
simulation you can do that as well.  This section provides an overview
of the stochastic controls available in BioSim.
The type {\tt StochasticIntensity} includes the following values:

\begin{itemize}
\item {\tt HIGH\_STOCH}
\item {\tt MEDIUM\_STOCH}
\item {\tt LOW\_STOCH}
\item {\tt NONE\_STOCH}
\end{itemize}

Each of these controls the width of the Gaussian\index{Gaussian}
function that provides the variations in output.  Setting the
stochastic intensity to NONE\_STOCH means that the simulation is
deterministic.  All noise is Gaussian.  The outputs of a module as
determined by the module's underlying equations are run through a
Gaussian filter before being output by the simulation.  The Gaussian
filter's parameters are controlled by the stochastic intensity, namely 
the higher the stochastic intensity, the higher the deviation.  

\subsection{Setting logging via the XML configuration file}
To change the stochastic intensity of a module in XML, simply add
setStochasticIntensity as an attribute to the module (if one isn't listed, it defaults to NONE\_STOCH).
example: <AirRS name="AirRS" setStochasticIntensity="HIGH\_STOCH">

The stochastic level for the entire simulation (which is equivalent to
setting each module's stochastic intensity) can by adding setStochasticIntensity to the Globals section at the head
of the XML document.
example:
<Globals crewsToWatch="CrewGroup" setStochasticIntensity="LOW\_STOCH">

\subsection{Setting logging via the API}
Each module of BioSim (i.e., the {\tt SimBioModule} class) has the
following methods:

\begin{itemize}
\item {\tt getStochasticIntensity()}: returns type {\tt
    StochasticIntensity}, which is the current stochastic level of the module.
\item {\tt setStochasticIntensity(StochasticIntensity pIntensity)}:
  Sets the current stochastic intensity of the module to {\tt
  pIntensity}.
\end{itemize}
The stochastic level for the entire simulation (which is equivalent to
setting each module's stochastic intensity) can be set via the {\tt
BioDriver} class using the {\tt setStochasticIntensity} method with
the same parameter ({\tt (StochasticIntensity})

\section{Logging}
\index{logging}

\subsection{Setting logging via the XML configuration file}

To change the logging of a module in XML, add
isLoggingEnabled as an attribute to the module (if one isn't listed, it defaults to false).
example: <AirRS name="AirRS" isLoggingEnabled="true">
You can also globally enable logging for the senors, actuators, or everything by adding
isFullLoggingEnabled, isSensorLoggingEnabled or isActuatorLoggingEnabled to the Globals section at the head
of the XML document.
example:
<Globals crewsToWatch="CrewGroup" isFullLoggingEnabled="true">

\subsection{Setting logging via the API}

Each module of BioSim (i.e., the {\tt SimBioModule} class) has the
following methods:

\begin{itemize}
\item {\tt isLogging()}: returns type {\tt
    boolean}, whether the module is logging.
\item {\tt setLogging(boolean pLogging)}:
  Sets whether the module should log.
\end{itemize}

\begin{figure}
\insertfigure{controller}{4.0in}
\caption{Interaction with the simulation.}
\label{fig:controller}
\end{figure}

\chapter{Controlling the simulation}
The previous two chapters showed you how to run BioSim, monitor its
internal state via a graphical user interface and configure it to your
specifications.  While interesting, this is not useful for control
system research.  This chapter will show you how to connect an
external control program to BioSim and change its underlying
operation.  Note that this chapter will not present an exhaustive list
of all of the objects and methods in BioSim, but only the most
important for new users.  Complete documentation of all objects and
method calls is available at:
http://www.traclabs.com/biosim/doc/api/index.html

Figure~\ref{fig:controller} shows the basic control strategy for the
simulation.  The simulation models represent the underlying equations
that govern the producer/consumer relationships (see
Section~\ref{sec:pc}).  Sensors and actuators are provided to access
the underlying simulations.  The user is responsible for writing the
controller, which uses the sensors and actuators to control the
simulation.  The user can also inject malfunctions into the underlying
simulation models and resupply consumables.   

\section{Controlling simulation runs}
The root class of the simulation is the {\tt BioDriver} class, which
contains the methods used to control the simulation.  The basic
control unit is a {\bf tick}\index{tick}, which advances the
simulation exactly one hour.  After creating your own BioDriver the
following methods are available:

\begin{enumerate}
\item  {\tt void advanceOneTick()}: Moves all modules of the
  simulation ahead exactly one hour.
\item {\tt void endSimulation()}: Ends\index{end} the simulation entirely.
\item {\tt boolean isDone()}: \index{isDone} Returns true if the simulation has met
  an end condition and has stopped.
\item {\tt void reset() }:\index{reset} Returns the simulation to its initial
  state.
\end{enumerate}

Here's an example of using one of these methods after a BioDriver
class has been created (called {\tt myBioDriver}) in a Java program:

\begin{verbatim}

    myBioDriver.advanceOneTick();

\end{verbatim}

There are a number of other methods available in the BioDriver class
that are less commonly used than those listed above.  These are
available via the JavaDoc\index{JavaDoc} facility.  After you have
installed your source code you can make documentation via the {\tt
  make-docs.sh} (for Linux) or {\tt make-docs} (for Windows)
executables in the {\tt bin} directory.  The BioDriverImpl class is
under the biosim.server.framework package.

\section{Accessing sensors and actuators}

\subsection{Generic sensors}

Sensors\index{sensors} are objects that connect to the underlying
simulation.  Each module has its own set of sensor objects, but they
all inherit the same method calls.  These method calls are:

\begin{itemize}

        \item {\tt getMax:} Returns the maximum allowed value of the sensor
        \item {\tt getMin:} Returns the minimum allowed value of the sensor
        \item {\tt getValue}: Returns the current setting of the sensor
\end{itemize}


\subsubsection{Additional sensor methods}

Several other generic methods related to sensor noise, malfunctions
and logging are available:

In addition, there will be some sensor-specific methods, which we will discuss with the appropriate sensor.

\subsection{Generic actuators}

Actuators\index{actuators} are objects that connect to the underlying simulation.  Each
module has its own set of actuator objects, but they all inherit the
same method calls.  These method calls are:

\begin{itemize}

        \item {\tt getMax}: Returns the maximum allowed value of the actuator
        \item {\tt getMin}: Returns the minimum allowed value of the actuator
        \item {\tt getValue}: Returns the current value of the actuator       
        \item {\tt setValue}: Sets the actuator

\end{itemize}

\subsubsection{Additional actuator methods}

Several other generic methods related to actuator noise, malfunctions
and logging are available:

In addition, there will be some actuator-specific methods, which we
will discuss with the appropriate actuator.


\subsection{Environments}

Environments\index{environment} contain the air\index{air} that crew
and crops use.  As there can be multiple environments in the simulation,
there can be multiple sensors tied to each environment.

\subsubsection{Sensors}
\label{sec:env_sen}
The following sensors are available in the environment:

\begin{enumerate}

\item {\tt AirInFlowRateSensor}: Flow rate in moles/tick of air into the environment
\item {\tt AirOutFlowRateSensor}: Flow rate in moles/tick of air out of the environment
\item {\tt CO2AirConcentrationSensor}: Percentage of CO2 in the environment.
\item {\tt CO2AirStoreInFlowRateSensor}: Flow rate in liters/tick of CO2 from a store. (paired with CO2AirEnvironmentInFlowRateSensor)
\item {\tt CO2AirEnvironmentInFlowRateSensor}: Flow rate in liters/tick of CO2 into the environment (paired with CO2AirStoreInFlowRateSensor)
\item {\tt CO2AirPressureSensor}: Pressure in kPA of CO2 in the environment
\item {\tt CO2AirStoreOutFlowRateSensor}: Flow rate in liters/tick of CO2 from a store. (paired with CO2AirEnvironmentOutFlowRateSensor)
\item {\tt CO2AirEnvironmentOutFlowRateSensor}: Flow rate in liters/tick of CO2 out of the environment (paired with CO2AirStoreOutFlowRateSensor)
\item {\tt NitrogenAirConcentrationSensor}: Percentage of Nitrogen in the environment
\item {\tt NitrogenAirEnvironmentInFlowRateSensor}: Flow rate in liters/tick of nitrogen into the environment (paired with NitrogenAirStoreInFlowRateSensor)
\item {\tt NitrogenAirStoreInFlowRateSensor}: Flow rate in liters/tick of nitrogen into a store (paired with NitrogenAirEnvironmentInFlowRateSensor)
\item {\tt NitrogenAirEnvironmentOutFlowRateSensor}: Flow rate in liters/tick of nitrogen out of an environment (paired with NitrogenAirStoreOutFlowRateSensor)
\item {\tt NitrogenAirStoreOutFlowRateSensor}: Flow rate in liters/tick of nitrogen out of a store (paired with NitrogenAirEnvironmentOutFlowRateSensor)
\item {\tt NitrogenAirPressureSensor}: Pressure in kPA of nitrogen in the environment
\item {\tt O2AirConcentrationSensor}: Percentage of O2 in the environment.
\item {\tt O2AirEnvironmentInFlowRateSensor}: Flow rate in liters/tick of O2 into the environment (paired with O2AirStoreInFlowRateSensor)
\item {\tt O2AirStoreInFlowRateSensor}: Flow rate in liters/tick of O2 out of the store (paired with O2AirEnvironmentInFlowRateSensor)
\item {\tt O2AirEnvironmentOutFlowRateSensor}: Flow rate in liters/tick of O2 out of the environment (paired with O2AirStoreOutFlowRateSensor)
\item {\tt O2AirStoreOutFlowRateSensor}: Flow rate in liters/tick of O2 out of the store (paired with O2AirEnvironmentOutFlowRateSensor)
\item {\tt O2AirPressureSensor}: Pressure in kPA of O2 in the environment
\item {\tt OtherAirConcentrationSensor}: Percentage of trace gases in the environment.
\item {\tt OtherAirPressureSensor}: Pressure in kPA of trace gases in the environment
\item {\tt WaterAirConcentrationSensor}: Percentage of water (humidity) in the environment
\item {\tt WaterAirEnvironmentInFlowRateSensor}: Flow rate in liters/tick of water vapor into the environment (paired with WaterAirStoreInFlowRateSensor)
\item {\tt WaterAirStoreInFlowRateSensor}: Flow rate in liters/tick of water vapor into the store (paired with WaterAirEnvironmentInFlowRateSensor)
\item {\tt WaterAirEnvironmentOutFlowRateSensor}: Flow rate in liters/tick of water vapor out of the environment (paired with WaterAirStoreOutFlowRateSensor)
\item {\tt WaterAirStoreOutFlowRateSensor}: Flow rate in liters/tick of water vapor out of the store (paired with WaterAirEnvironmentOutFlowRateSensor)
\item {\tt WaterAirPressureSensor}: Pressure in kPA of water in the environment
\end{enumerate}

\subsubsection{Actuators}

\begin{enumerate}

\item {\tt AirInFlowRateActuator}: Flow rate in moles/tick of air into the environment
\item {\tt AirOutFlowRateActuator}: Flow rate in moles/tick of air out of the environment
\item {\tt CO2AirStoreInFlowRateActuator}: Flow rate in liters/tick of CO2 from a store. (paired with CO2AirEnvironmentInFlowRateActuator)
\item {\tt CO2AirEnvironmentInFlowRateActuator}: Flow rate in liters/tick of CO2 into the environment (paired with CO2AirStoreInFlowRateActuator)
\item {\tt CO2AirStoreOutFlowRateActuator}: Flow rate in liters/tick of CO2 from a store. (paired with CO2AirEnvironmentOutFlowRateActuator)
\item {\tt CO2AirEnvironmentOutFlowRateActuator}: Flow rate in liters/tick of CO2 out of the environment (paired with CO2AirStoreOutFlowRateActuator)
\item {\tt NitrogenAirEnvironmentInFlowRateActuator}: Flow rate in liters/tick of nitrogen into the environment (paired with NitrogenAirStoreInFlowRateActuator)
\item {\tt NitrogenAirStoreInFlowRateActuator}: Flow rate in liters/tick of nitrogen into a store (paired with NitrogenAirEnvironmentInFlowRateActuator)
\item {\tt NitrogenAirEnvironmentOutFlowRateActuator}: Flow rate in liters/tick of nitrogen out of an environment (paired with NitrogenAirStoreOutFlowRateActuator)
\item {\tt NitrogenAirStoreOutFlowRateActuator}: Flow rate in liters/tick of nitrogen out of a store (paired with NitrogenAirEnvironmentOutFlowRateActuator)
\item {\tt O2AirEnvironmentInFlowRateActuator}: Flow rate in liters/tick of O2 into the environment (paired with O2AirStoreInFlowRateActuator)
\item {\tt O2AirStoreInFlowRateActuator}: Flow rate in liters/tick of O2 out of the store (paired with O2AirEnvironmentInFlowRateActuator)
\item {\tt O2AirEnvironmentOutFlowRateActuator}: Flow rate in liters/tick of O2 out of the environment (paired with O2AirStoreOutFlowRateActuator)
\item {\tt O2AirStoreOutFlowRateActuator}: Flow rate in liters/tick of O2 out of the store (paired with O2AirEnvironmentOutFlowRateActuator)
\item {\tt WaterAirEnvironmentInFlowRateActuator}: Flow rate in liters/tick of water vapor into the environment (paired with WaterAirStoreInFlowRateActuator)
\item {\tt WaterAirStoreInFlowRateActuator}: Flow rate in liters/tick of water vapor into the store (paired with WaterAirEnvironmentInFlowRateActuator)
\item {\tt WaterAirEnvironmentOutFlowRateActuator}: Flow rate in liters/tick of water vapor out of the environment (paired with WaterAirStoreOutFlowRateActuator)
\item {\tt WaterAirStoreOutFlowRateActuator}: Flow rate in liters/tick of water vapor out of the store (paired with WaterAirEnvironmentOutFlowRateActuator)

\end{enumerate}

\subsubsection{Malfunctions}
State exactly what the malfunctions do.  How do the method calls work.
What is the malfunction. What is the leak rate, etc. for each of the
different malfunction classes.  Since method calls are probably
identical may want to move this up to top and then use this section
only for module-specific malfunction information.

\subsection{Crew}
\index{crew}
\subsubsection{Sensors}

\begin{enumerate}

\item {\tt CrewGroupActivitySensor}:  Returns average activity level of crew
\item {\tt CrewGroupAnyDeadSensor}: Returns whether any crew member has died
\item {\tt CrewGroupDeathSensor}: Return whether the entire crew has died
\item {\tt CrewGroupProductivitySensor}: Total productivity of the crew

\end{enumerate}

\subsubsection{Actuators}

\begin{enumerate}

\item {\tt CrewGroupActivityActuator}:

\end{enumerate}

\subsubsection{Malfunctions}

\subsection{Air}
\index{air}
\subsubsection{Sensors}

\begin{enumerate}
\item {\tt CO2InFlowRateSensor}: Reports the CO2 flow rate (in moles) into a module
\item {\tt CO2OutFlowRateSensor}: Reports the CO2 flow rate (in moles) out of a module
\item {\tt CO2StoreLevelSensor}: How much CO2 (in moles) is contained in this store.
\item {\tt H2InFlowRateSensor}: Reports the H2 flow rate (in moles) into a module
\item {\tt H2OutFlowRateSensor}: Reports the H2 flow rate (in moles) out of a module
\item {\tt H2StoreLevelSensor}: How much   (in moles) is contained in this store.
\item {\tt NitrogenInFlowRateSensor}: Reports the nitrogen flow rate (in moles) into a module
\item {\tt NitrogenOutFlowRateSensor}: Reports the nitrogen flow rate (in moles) out of a module
\item {\tt NitrogenStoreLevelSensor}: How much nitrogen (in moles) is contained in this store.
\item {\tt O2InFlowRateSensor}: Reports the O2 flow rate (in moles) into a module
\item {\tt O2OutFlowRateSensor}: Reports the O2 flow rate (in moles) out of a module
\item {\tt O2StoreLevelSensor}: How much O2 (in moles) is contained in this store.
\end{enumerate}

\subsubsection{Actuators}

\begin{enumerate}
\item {\tt CO2InFlowRateAcutator}: Changes how much CO2 (in moles) goes into the module
\item {\tt CO2OutFlowRateAcutator}: Changes how much CO2 (in moles) goes out of the module
\item {\tt H2InFlowRateAcutator}: Changes how much H2 (in moles) goes into the module
\item {\tt H2OutFlowRateAcutator}: Changes how much H2 (in moles) goes out of the module
\item {\tt NitrogenInFlowRateAcutator}: Changes how much nitrogen (in moles) goes into the module
\item {\tt NitrogenOutFlowRateAcutator}: Changes how much nitrogen (in moles) goes out of the module
\item {\tt O2InFlowRateAcutator}: Changes how much O2 (in moles) goes into the module
\item {\tt O2OutFlowRateAcutator}: Changes how much O2 (in moles) goes out of the module
\end{enumerate}

\subsubsection{Malfunctions}

Same as others


\subsection{Water}
\index{water}

\subsubsection{Sensors}
\begin{enumerate}
\item {\tt DirtyWaterInFlowRateSensor}: Reports the dirty water flow rate (in liters) into a module
\item {\tt DirtyWaterOutFlowRateSensor}: Reports the dirty water flow rate (in liters) out of a module
\item {\tt DirtyWaterStoreLevelSensor}: How much dirty water (in liters) is contained in this store.
\item {\tt GreyWaterInFlowRateSensor}: Reports the grey water flow rate (in liters) into a module
\item {\tt GreyWaterOutFlowRateSensor}: Reports the grey water flow rate (in liters) out of a module
\item {\tt GreyWaterStoreLevelSensor}: How much grey water (in liters) is contained in this store.
\item {\tt PotableWaterInFlowRateSensor}: Reports the potable water flow rate (in liters) into a module
\item {\tt PotableWaterOutFlowRateSensor}: Reports the potable water flow rate (in liters) out of a module
\item {\tt PotableWaterStoreLevelSensor}: How much potable water (in liters) is contained in this store.
\end{enumerate}

\subsubsection{Actuators}

\begin{enumerate}
\item {\tt DirtyWaterInFlowRateActuator}: Changes how much dirty water (in liters) goes into the module
\item {\tt DirtyWaterOutFlowRateActuator}: Changes how much dirty water (in liters) goes ouf of the module
\item {\tt GreyWaterInFlowRateActuator}: Changes how much grey water (in liters) goes into the module
\item {\tt GreyWaterOutFlowRateActuator}: Changes how much grey water (in liters) goes ouf of the module
\item {\tt PotableWaterInFlowRateActuator}: Changes how much potable water (in liters) goes into the module
\item {\tt PotableWaterOutFlowRateActuator}: Changes how much potable water (in liters) goes ouf of the module
\end{enumerate}

\subsubsection{Malfunctions}

\subsection{Biomass}
\index{biomass}

\subsubsection{Sensors}
\begin{enumerate}
\item {\tt BiomassInFlowRateSensor}: Reports the biomass flow rate (in kilograms) into a module
\item {\tt BiomassOutFlowRateSensor}: Reports the biomass flow rate (in kilograms) out of a module
\item {\tt BiomassStoreLevelSensor}: How much biomass (in kilograms) is contained in this store.
\item {\tt FoodInFlowRateSensor}: Reports the food flow rate (in kilograms) into a module
\item {\tt FoodOutFlowRateSensor}: Reports the food flow rate (in kilograms) out of a module
\item {\tt FoodStoreLevelSensor}: How much food (in kilograms) is contained in this store.
\end{enumerate}

\subsubsection{Actuators}
\begin{enumerate}
\item {\tt BiomassInFlowRateActuator}: Changes how much biomass (in kilograms) goes into the module
\item {\tt BiomassOutFlowRateActuator}: Changes how much biomass (in kilograms) goes out of the module
\item {\tt FoodInFlowRateActuator}: Changes how much food (in kilograms) goes into the module
\item {\tt FoodOutFlowRateActuator}: Changes how much food (in kilograms) goes out of the module
\end{enumerate}

\subsubsection{Malfunctions}

\subsection{Dry Waste}
\index{waste}
\subsubsection{Sensors}
\begin{enumerate}
\item {\tt DryWasteInFlowRateSensor}: Changes how much dry Waste (in kilograms) goes into the module
\item {\tt DryWasteOutFlowRateSensor}: Changes how much dry Waste (in kilograms) goes out of the module
\item {\tt DryWasteStoreLevelSensor}: How much dry Waste (in kilograms) is contained in this store.
\end{enumerate}

\subsubsection{Actuators}
\begin{enumerate}
\item {\tt DryWasteInFlowRateActuator}: Changes how much dry waste (in kilograms) goes into the module
\item {\tt DryWasteOutFlowRateActuator}: Changes how much dry waste (in kilograms) goes out of the the module
\end{enumerate}

\subsubsection{Malfunctions}

\subsection{Power}
\index{power}

\subsubsection{Sensors}

\begin{enumerate}
\item {\tt PowerInFlowRateSensor}: Changes how much power (in watts) goes into the module
\item {\tt PowerOutFlowRateSensor}: Changes how much power (in watts) goes out of the module
\item {\tt PowerStoreLevelSensor}: How much power (in watts) is contained in this store.
\end{enumerate}

\subsubsection{Actuators}

\begin{enumerate}
\item {\tt PowerInFlowRateActuator}: Changes how much power (in watts) goes into the module
\item {\tt PowerOutFlowRateActuator}: Changes how much power (in watts) goes out of the module
\end{enumerate}

\subsubsection{Malfunctions}



\chapter{Writing a simple Java controller}

A short tutorial on creating a BioSim, initializing it and then
calling some sensors, actuators and ticks.  Assume JacOrb. 
See BIOSIM/doc/TestBiosim.java

\chapter{Writing a simple C++ controller}



Same as above in C++.  Assume ACEOrb.  Assume Linux (?).

\chapter{Control Examples}

Include GA, Rice learners, Texas Tech learner.

\chapter{Conclusions}

\appendix 
\chapter{Java Example}

\begin{verbatim}

Put parts of Rice handcontroller here

\end{verbatim}

\chapter{C++ Example}

\begin{verbatim}

#include "biosimC.h"
#include <CosNamingC.h>

static const char* CorbaServerName = "BioDriver0";
int
main(int argc,char *argv[])
{
  try {
    // Initialize orb
    CORBA::ORB_var orb = CORBA::ORB_init(argc,argv,"TestClient");

    cerr << "initialized orb" << endl;

    if (argc < 2) {
      cerr << "Usage: " << argv[0]
           << " IOR_string" << endl;
      return 1;
    }

// Use the first argument to create the object reference,
    // in real applications we use the naming service, but let's do
    // the easy part first!

    CORBA::Object_var obj =
      orb->string_to_object (argv[1]); 

    cout << "After resolve_initial_references" << endl;


    /*   CORBA::Object_var obj =
         orb->string_to_object (argv[1]); */
 
    CosNaming::NamingContext_var inc = CosNaming::NamingContext::_narrow(obj);

    cout << "After NameService narrow" << endl;

    if(CORBA::is_nil(inc)) {
      cerr << "Unable to narrow NameService." << endl;
      throw 0;
    }  // end if couldn't get root naming context
    
    CosNaming::Name name;
    name.length(1);
    name[0].id = CORBA::string_dup(CorbaServerName);
    name[0].kind = CORBA::string_dup("");

    CORBA::Object_var tmp = inc->resolve(name);

    cout << "After resolve name" << endl;

    if(CORBA::is_nil(tmp)) {
      cerr << "Unable to resolve " << CorbaServerName << " from NameService" << endl;
      throw 0;
    }  // end if couldn't find object

    cerr << "got object" << endl;

    // Now downcast the object reference to the appropriate type

    biosim::idl::framework::BioDriver_var biodriver = biosim::idl::framework::BioDriver::_narrow(tmp);


    cerr << "narrowed" << endl;

    // code here

    biodriver->spawnSimulationTillDead();

    cerr << "spawned" << endl;

    orb->destroy ();
  }
  catch (CORBA::Exception &ex) {
    std::cerr << "CORBA exception raised! " << std::endl;
  }
  return 0;
}

\end{verbatim}

\bibliography{biosim}
\bibliographystyle{plain}

\begin{theindex}

\documentstyle{report}

\input epsf

%\nofiles

\setlength{\textheight}{9.0in}
\setlength{\textwidth}{6.5in}
\setlength{\footheight}{0.0in}
\setlength{\topmargin}{0.00in}
\setlength{\headheight}{0.0in}
\setlength{\headsep}{0.0in}
\setlength{\oddsidemargin}{-0.0in}

\makeindex

\begin{document}

\def\topfraction{1.0}        %maximum fraction of floats at the top of the page
\def\bottomfraction{1.0}     %ditto, for the bottom of the page
\def\textfraction{0}         %minimum fraction of text (--> 100% floats is okay)
\def\floatpagefraction{1.0}  % if a page is full of floats, it'd better be FULL

\newcommand{\insertfigure}[2]{\begin{center}
                             \ \epsfxsize=#2 \epsfbox{#1.ps}
                             \end{center}}

%\newcommand{\insertfigure}[2]{\vspace{#2}}

\title{BioSim: An Integrated Simulation of an Advanced Life Support System for Intelligent Control Research\\
Users Manual (draft)}

\author{Metrica Inc. and S\&K Technologies\\
NASA Johnson Space Center/ER2\\
Houston TX 77058\\
http://www.traclabs.com/biosim}


\date{\today}

\maketitle

\tableofcontents

\newpage

\listoffigures

\newpage

\chapter{Introduction}
Advanced life support systems have multiple interacting subsystems,
which makes control a particularly challenging task.  The simulation
described in this document provides a testbed for integrated control
research.  There have been other integrated life support simulations
(e.g., \cite{finn99}) and we have learned from those efforts.  This
simulation is designed exclusively for integrated controls research,
which imposes different requirements.  For example, the simulation is
accessed through sensors and actuators, just as a real system would
be.  Noise and uncertainty are built in and controllable.
Malfunctions and failures of subsystems are modeled and manifest
themselves through anomalous readings in the sensors.  Crew members
are taskable and their tasks have purpose and meaning in the
simulation.  In essence, the simulation is a replacement for the
Advanced Life Support (ALS) hardware and crew, allowing for testing of
control approaches in advance of any integrated test.  We want the
simulation to be used to develop and evaluate integrated control
techniques.  There are still many open research questions with respect
to controlling advanced life support systems.  For example is a
distributed or hierarchical approach better?  What role does machine
learning play in control?  How can symbolic, qualitative control
approaches be integrated with continuous, quantitative approaches?
How can we evaluate different control philosophies?  We hope that the
controls community can use this simulation to begin to build systems
which will provide us with answers to these kinds of questions. 

The simulation is written in Java and is accessed using the Common
Object Request Broker Architecture (CORBA)\index{CORBA}.  The distributed nature of
the simulation allows for multiple instances to be run in parallel,
making it ideal for testing advanced control concepts such as genetic
algorithms or reinforcement learning, which require multiple trials.
See \cite{kortenkamp_etal01} or \cite{schreckenghost_etal02} for a
background on advanced life support control issues and previous work.
This document begins with a brief introduction to advanced life
support systems and provides details on installing, running and
interacting with the simulation.  The simulation is available from:

\begin{verbatim}
     http://www.traclabs.com/biosim
\end{verbatim}

\chapter{Background on advanced life support}

An advanced life support system is one in which many resources are
reused or regenerated.  Such systems will be necessary for long
duration human missions, such as those to Mars, where resupply
opportunities are limited.  Typically they have thin margins for mass,
power and buffers, which requires optimization and tight control.
Also, advanced life support systems consist of many interconnected
subsystems all of which interact in both predictable and unpredictable
ways (see Figure~\ref{fig:biosim}).  Autonomous control\index{control}
of these systems is desirable. This section gives enough background so
that users of the simulation can understand the major components of an
advanced life support system.  For detailed documentation on advanced
life support systems see \cite{tri99} or go to:

\begin{verbatim}
    http://advlifesupport.jsc.nasa.gov
\end{verbatim}

\begin{figure}
\insertfigure{figure}{5.5in}
\caption{The various modules that comprise the BIOSim integrated simulation.}
\label{fig:biosim}
\end{figure}

\section{Modules}

An advanced life support system consists of many interacting
subsystems.  While each is self-contained, they rely on each other for
various resources.  The simulation is built from a set of modules that
each produce and consume different resources.

\subsection{Environment}
\label{sec:env}
An environment\index{environment} contains air that is consumed by
either people or crops.  Air\index{air} contains a mixture of gases --
in our simulation these gases are oxygen (O2), carbon dioxide (CO2),
nitrogen (N), water vapor (H20), and other gasses (trace).  The initial composition of the
gases is set by the simulation initialization file (see
Section~\ref{sec:init}).  As the simulation runs modules may consume
air from the simulation and replace it with air of a different
composition.  Thus, the composition of gases and pressure in the air change over
time and can be measured by environment sensors
(Section~\ref{sec:env_sen}).  As with all modules, there can be
multiple environments.  For example, it is common for crew members
and crops to have different air compositions and that is the default
in this simulation.   

\subsection{Crew}
\label{sec:crew}

The crew\index{crew} module is implemented using models described in
\cite{goudarzi_ting99}.  The number, gender, age and weight of the
crew are settable as input parameters -- in the default configuration
there are four crew members, two male and two female.  The crew cycles
through a set of activities (sleep, maintenance, recreation, etc.).
As they do so they consume O2, food and water and produce CO2, dirty
water and solid waste.  The amount of resources consumed and produced
varies according to crew member attributes and their activities.  The
crew's activities can be adjusted by passing a new crew schedule to
the crew module.  A default schedule can also be used.  The crew
module is connected to a crew environment that contains an atmosphere
that they breathe.  The initial size and gas composition (percentages
of O2, CO2, H2O and inert gases) are input parameters and the default
is $1.54893 x 10^6 liters$ (from \cite{tri99}) with an atmosphere
equivalent to sea level air.  As the simulation progresses the mixture
of gases in the atmosphere changes.

\subsection{Water}
\label{sec:water}
Consumes:Power, Grey Water, Dirty Water
Produces: Potable Water

The water\index{water} recovery module consumes dirty water, grey water(i.e., water that can be used for
washing but not drinking), and power and
produces potable water.  The water recovery module consists of four
subsystems that process the water.  The biological water processing
(BWP) subsystem removes organic compounds.  Then the water passes to a
reverse osmosis (RO) subsystem, which makes 85\% of the water passing
through it grey.  The 15\% of water remaining from the RO (called
brine) is passed to the air evaporation subsystem (AES), which
recovers the rest.  These two streams of grey water (from the RO and
the AES) are passed through a post-processing subsystem (PPS) to
create potable water.  An external controller can turn on or off
various subsystems.  For example, all water can pass through the AES
at a higher power cost.  We based our water recovery module on a
recently completed test at NASA Johnson Space Center
\cite{bonasso_etal03}.

\subsection{Air}
\label{sec:air}
Consumes:Power, H2, Air, CO2, Potable Water
Produces: Air (with less CO2), O2, H2, CO2, Potable Water

The air\index{air} component takes in exhalant CO2 and produces O2 as long as
there is sufficient power being provided to the system.  This
component is modeled on various Air Revitalization System (ARS) work
at NASA JSC \cite{malin_etal00}.  There are three interacting air
subsystems: the Variable Configuration Carbon Dioxide Removal (VCCR)
System in which CO2 is removed from the air stream; the Carbon Dioxide
Reduction System (CRS), which also removes CO2 from the air stream
using a different process and producing different gases than the VCCR;
and the Oxygen Generation System (OGS) in which O2 is added to the air
stream by breaking water down into hydrogen and oxygen.  It is
important to note that both the removal of CO2 and the addition of O2
are required for human survival.  It is also important to note that
the biomass component (next subsection) also removes CO2 and adds O2.

\subsection{Biomass}
\label{sec:biomass}
Consumes:Power, Potable Water, Grey Water, Air
Produces: Air (with more CO2), Biomass, Dirty Water, CO2, Potable Water

The biomass\index{biomass} component is where crops\index{crops} are
grown.  It produces both biomass, which can be turned into
food\index{food}, and O2 and consumes water, power (light) and CO2.
This component is based on models given in \cite{jones_cavazzoni00}.
The system is modeled as shelves\index{shelves} that contain plants,
lights and water.  Shelves are planted and harvested and there is
growth cycle for each shelf.  Currently, ten crops are modeled and can
be planted in any ratio.

\subsection{Food processing}
\label{sec:food}
Consumes:Power, Biomass
Produces: Food

Before biomass can be consumed by the crew it must be converted to
food\index{food}.  The food processing component takes biomass, power and crew
time and produces food and solid waste.  The crew needs to be involved
in this process as it is labor intensive.  See Section~\ref{sec:crew}
for more information on scheduling crew activities.

\subsection{Waste}
\label{sec:waste}
Consumes:Power, Dry Waste, O2
Produces: CO2

The waste\index{waste} component consumes power, O2 and solid waste
and produces CO2.  It is modeled on an incinerator used in the Phase
III test in 1997 \cite{schreckenghost_etal02}.  Incineration can be
scheduled.

\subsection{Thermal control}
\label{sec:thermal}
Consumes: ?
Produces: ?

The thermal control\index{thermal} component regulates the air temperature in the
habitat.  This component is not implemented for this simulation and it
is assumed that external controllers are maintaining chamber
temperature.

\subsection{Power}
\label{sec:power}
Consumes: 
Produces: Power

The power\index{power} component supplies power to all of the other
components that need it.  There are two choices for power in the
simulation.  The first is nuclear\index{nuclear} power, which supplies
a fixed amount throughout the lifetime of the simulation.  The second
is solar\index{solar} power, which supplies a varying amount
(day/night cycle) of power to each component.

\subsection{Accumulators}
\label{sec:accumulator}
Consumes: All
Produces: All

The accumulator\index{accumulator} component can take a resource from any
store or environment and place it into another environment or store.  It is
functionally equivalent to an injector.

\subsection{Injectors}
\label{sec:injector}
Consumes: All
Produces: All

The injector\index{injector} component can take a resource from any
store or environment and place it into another environment or store.  It is
functionally equivalent to an accumulator.

\chapter{Simulation Properties}

The simulation implements the modules outlined in the previous
section.  BioSim does {\em not} simulate at the level of valves,
pumps, switches, etc.  Instead, modules are implemented in a
producer/consumer relationship, which is described in the next
subsection.  The simulation also provides additional functionality to
help test and debug control programs.

\section{Producer/Consumer Model}

No component in BioSim directly interacts with other components.  Instead, each component
has a rigid set of resources that it consumes and produces.  The resources are taken from stores/environments
and put into stores/environments.  The resources for BioSim currently consist of
power, potable water, grey water, dirty water, air, H2, nitrogen, O2, CO2, biomass, food, and dry waste.  At each simulation tick,
each module takes resource from it's store (it's consumables), and puts resources into stores (it's products).  Stores always report
thier values (level and capacity) from the last tick and report new values after every component in the simulation has been ticked.  
Resource conflicts can occur when two components ask for a limited resource, i.e. two WaterRS's both need 20 liters of dirty water and there
is only 20 liters in the dirty water tank.  This is currently resolved on a winner-take-all basis that is for all practical purposes
randomized.

\section{Success criteria}

In order to compare different control approaches there should be
objective criteria for a successful advanced life support system
mission.  Several possibilities exist already in the simulation.  For
example, the length of the mission before consumables are gone is a
success criterion.  As is minimizing the starting levels of stores or
minimizing the sizes of intermediary buffers.  However, many of these
fail to get at the true success of a mission, the productivity of the
crew in performing their science objectives.  Thus, the simulation
includes an artificial productivity\index{productivity} measure.  As
the crew cycles through their activities (see Section~\ref{sec:activities})
the amount of time they spend doing "mission" tasks is accumulated. This
number is also multiplied by a factor that takes into account the
amount of sleep, exercise, water, food and oxygen they are getting to approximate 
crew effectivness at performing the task (a happy crew is a productive crew!).
So each tick, for each crew member the productivity is given by the
following equation:

        [need to put the equation here]
	[some relevant code, derive equation later]
	\begin{verbatim}
	private static final float WATER_TILL_DEAD = 8.1f;
	private static final float WATER_RECOVERY_RATE=0.01f;
	private static final float OXYGEN_TILL_DEAD = 3f;
	private static final float OXYGEN_RECOVERY_RATE=0.01f;
	private static final float CALORIE_TILL_DEAD = 180000f;
	private static final float CALORIE_RECOVERY_RATE=0.0001f;
	private static final float DANGEROUS_CO2_RATION = 0.06f;
	private static final float CO2_TILL_DEAD = 10f;
	private static final float CO2_RECOVERY_RATE=0.001f;
	private static final float LEISURE_TILL_BURNOUT = 168f;
	private static final float LEISURE_RECOVERY_RATE=90f;
	private static final float AWAKE_TILL_EXHAUSTION = 120f;
	private static final float SLEEP_RECOVERY_RATE=12f;
	consumedWaterBuffer = new SimpleBuffer(WATER_TILL_DEAD, WATER_TILL_DEAD);
	consumedOxygenBuffer = new SimpleBuffer(OXYGEN_TILL_DEAD, OXYGEN_TILL_DEAD);
	consumedCaloriesBuffer = new SimpleBuffer(CALORIE_TILL_DEAD, CALORIE_TILL_DEAD);
	consumedCO2Buffer = new SimpleBuffer(CO2_TILL_DEAD, CO2_TILL_DEAD);
	sleepBuffer = new SimpleBuffer(AWAKE_TILL_EXHAUSTION, AWAKE_TILL_EXHAUSTION);
	leisureBuffer = new SimpleBuffer(LEISURE_TILL_BURNOUT, LEISURE_TILL_BURNOUT);
	float caloriePercentFull = sigmoidLikeProbability(consumedCaloriesBuffer.getLevel() / consumedCaloriesBuffer.getCapacity());
	float waterPercentFull = sigmoidLikeProbability(consumedWaterBuffer.getLevel() / consumedWaterBuffer.getCapacity());
	float oxygenPercentFull = sigmoidLikeProbability(consumedOxygenBuffer.getLevel() / consumedOxygenBuffer.getCapacity());
	float CO2PercentFull = sigmoidLikeProbability(consumedCO2Buffer.getLevel() / consumedCO2Buffer.getCapacity());
	float sleepPercentFull = sleepBuffer.getLevel() / sleepBuffer.getCapacity();
	float leisurePercentFull = leisureBuffer.getLevel() / leisureBuffer.getCapacity();
	float averagePercentFull = (caloriePercentFull + waterPercentFull + oxygenPercentFull + CO2PercentFull + sleepPercentFull + leisurePercentFull) / 6f;
	myMissionProductivity += myCrewGroup.randomFilter(averagePercentFull);
	\end{verbatim}
\section{Stochastic processes}

Any sufficiently complex process, especially one with biological
components, will not be deterministic\index{deterministic}.  That is,
given the same starting conditions and inputs it will not produce the
exact same outputs each time it is run.  A
stochastic\index{stochastic} process is one in which chance or
probability affect the outcome.  This simulation offers both
deterministic and stochastic operations.  In the former case, running
the simulation twice with the same inputs and initial conditions will
produce the same results.  This may be useful for quantitatively
comparing different control approaches against each other.  In the
latter case, the user can control the amount of variability in the
simulation (see Section 7.4).  This is modeled using a Gaussian\index{Gaussian}
distribution.  The deviation is determined by the stochastic intensity; a higher intensity will yield a higher deviation.  The filter is then appropriately applied to certain variable of the model.  This can be used to test a
control system�s ability to deal with stochastic processes.

\section{Malfunctions}

The simulation also has the ability to accept
malfunction\index{malfunctions} requests (see Section 7.5).  These
requests will change the operating regime of the simulation.  For
example, by causing a crew member to be sick, power supplies to drop,
water to leak, plants to die, etc.  Different modules will have
different sensitivities, for example once plants die they cannot
recover, but if the water module is damaged and then repaired it will
operate normally.  Of course, if levels of CO2, water or food reach
hard-coded critical levels the crew will abandon the mission and return
home.

\section{Crew activity scheduling}
\label{sec:activities}
Crew members in an advanced life support system do not just consume
and produce -- they do activities.  These may be science, maintenance,
exercise, sleep, etc.  The different activities go for a specified length
and for a specified intensity.  The intensity is tied to how many resources the crew
member will consume performing the activity (e.g., sleep takes less 02 to perform than exercise).  
The following diagram shows the nominal daily routine for a crew member:

Schedule loop:
	Sleep for 8 hours
	Hygiene for 1 hour
	Exercise for 1 hour
	Eating for 1 hour
	Mission tasks for 8 hours
	Health for 1 hour
	Maintenance for 1 hour
	Leisure for 2 hours

This schedule can be interrupted by malfunctions (a crew member's activity changed to repair),
sickness, or even mission end (if the crew member hasn't recieved proper resources).

\section{Mass balance}
Mass balanced means resources consumed by components are returned in totality (but usually in a different form) to stores, no mass is lost.
This is important for closed loop simulations as it verfies the models (under perfect conditions) are keeping with the conversation of energy and mass.
Our simulation has been mass balanced to a reasonable degree.  There is a slight loss in mass from the crew (they don't gain weight from eating or produce heat)
and from the plants (who don't produce heat).  We plan to address these issues in the future, but the simulation is currently mass balanced
enough to provide a very decent closed loop simulation.


\chapter{Simulation implementation}

All of the simulation components are written in Java\index{Java} to
make the simulation portable.  Each component can be run as it's own
CORBA\index{CORBA} server so that they can be distributed across
computers for increased performance, or the entire simulation can be run on one server.  We have tested the simulation on
Windows\index{Windows}, Unix\index{Unix} and
Macintosh\index{Macintosh} computers.  A user interface allows for
insight into the simulation which provides real time graphing and more.  There is also a logging\index{logging}
facility that will automatically save all simulation values to a file (xml)
or a database (in development).  We have written Java and C++\index{C++} interfaces to the simulation, though any
language that supports CORBA should be able to access the simulation.

(BioSim) <-> Server ORB <-> Client ORB <-> (Client)

I think we should provide a link to a CORBA tutorial, but assume knowledge of basics of CORBA.
Explain object oriented methodology and how it relates to BioSim.
Would a figure be good here showing the relationship between servers,
clients, CORBA, etc.?

\chapter{Installing the simulation}
The simulation is written in Java and can run on most popular
operating systems.  We provide both an executable for Windows and
source code for any operating system.  This chapter deals with
installing the code onto your computer. 

\section{Installing an executable}

An executable that will run under most Windows operating systems is
available.  You can download the Windows executable from: {\tt
  www.traclabs.com/biosim}.

At that site, go to Download and then download the Windows
installer\index{installation}.  Remember where you put it.
Double-click on the Setup.exe icon, accept the agreement, choose the
default installation, install it where you desire and then click
``Finish''.  BIOSim will start automatically.  NOTE: If you do not
have Java\index{Java} installed on your PC then the installer will
direct you to the WWW page to download it (free).  Please e-mail Scott
Bell (scott@traclabs.com) if you have trouble with installing the
simulation.

\begin{figure}
\insertfigure{labelled_interface}{5.5in}
\caption{BioSim�s graphical user interface.}
\label{fig:empty_interface}
\end{figure}

\section{Installing source code}

Please read the README.txt file in the doc directory of your BIOSIM distribution.

\chapter{Running the simulation}
\label{sec:running}

Doubleclicking on the ``BioSim'' icon in Windows will start the
simulation.  In Linux, run the {\tt run-biosim.sh} shell script in the
{\tt /bin} directory.  When BioSim starts you'll see a blank
interface\index{interface} as shown in
Figure~\ref{fig:empty_interface}.  Holding your mouse over the buttons
on the top reveals what they do.  As noted in the figure, the first
three buttons control the simulation.  The next series of buttons will
display various subsystem views. Some of the views, like the crew, do
not contain anything until the simulation is started.  I like to open
up the main subsystem views before starting the simulation.

The open interface is shown in Figure~\ref{fig:full_interface}.  The
view can be changed between text, chart and schematic by clicking on
the tabs.  The simulation will run until resources (air, water, food,
etc.) drop below critical levels.  Then the simulation will stop.
Pressing the start button will restart the simulation from the
beginning.  If you want to examine the interfaces closely press the
``Pause'' button or go through the simulation one step at a time.

\begin{figure}
\insertfigure{open_interface}{5.5in}
\caption{A full view of BioSim�s graphical user interface}
\label{fig:full_interface}
\end{figure}

\chapter{Configuring the simulation}
The previous chapter discussed installing and running the default
BioSim.  This section will discuss configuring BioSim to address
specific user interests.  This includes changing the underlying
producer/consumer relationship, changing the stochastic nature of
BioSim and changing the logging of data.  We start with some basic
CORBA infrastructure requirements.     

\section{CORBA, Orbs and namesevers}

BioSim is a server.  Client programs connect to BioSim via
CORBA\index{CORBA} -- the Common Object Request Broker Architecture.
In order to connect to the BioSim server will you will need to install
an Object Request Broker (ORB)\index{ORB} on your machine.  We
recommend the JacORB for Java (http://www.jacorb.org) and the ACEOrb for C++ (http://www.cs.wustl.edu/~schmidt/TAO.html), 
although any CORBA-compliant ORB will work.  When the BioSim
server is started it registers with a nameserver\index{nameserver},
which tells client programs where to find it.  The ORBs we recommend
include a name server.  This manual will not cover the details of
CORBA, we refer you to any book on CORBA for your language or look at
the sample programs in Appendix A and B.  We also assume that you have
already installed one of the ORBs.
   
\section{Initial conditions}
\label{sec:init}
BioSim is almost infinitely reconfigurable\index{recongifuration} --
from the number of crew members to the size and number of the
different modules and should be able to provide scenarios ranging from quick shuttle launces to Mars colonies.  
The start-up configuration of BioSim is controlled
by an eXtensible Markup Language (XML)\index{XML} file that is read in
during BioSim initialization.  If you want to change BioSim's
configuration you need to change the XML file.  The default XML file
is located in:

\begin{verbatim}
   BIOSIM/resources/biosim/server/framework/DefaultInitialization.xml
\end{verbatim}

\noindent where {\tt BIOSIM} is the directory where you installed
BioSim.

You can copy this file, change it's name, modify it and then give its
full path name as a parameter to BioSim on startup.  The flag when you
start BioSim is {\tt -xml} followed by the full path name of the XML
file.  If you do not wish to change the initial configuration of
BioSim you can skip the rest of this section. 

\subsection{XML file format}
A snippet from the default initialization XML file is shown below:

\begin{verbatim}
  <SimBioModules>
    <crew>
      <CrewGroup name="CrewGroup">
        <potableWaterConsumer maxFlowRates="100" desiredFlowRates="100" inputs="PotableWaterStore"/>
        <airConsumer maxFlowRates="0" desiredFlowRates="0" inputs="CrewEnvironment"/>
        <foodConsumer maxFlowRates="100'' desiredFlowRates="100" inputs="FoodStore"/>
        <dirtyWaterProducer maxFlowRates="100" desiredFlowRates="100" outputs="DirtyWaterStore"/>
        <greyWaterProducer maxFlowRates="100" desiredFlowRates="100" outputs="GreyWaterStore"/>
        <airProducer maxFlowRates="0" desiredFlowRates="0" outputs="CrewEnvironment"/>
         .
         .
         .
        <crewPerson name="Bob Roberts" age="43" weight="77" sex="MALE">
          <schedule>
            <activity name="sleep" length="8" intensity="1"/>
            <activity name="hygiene" length="1" intensity="2"/>
            <activity name="exercise" length="1" intensity="5"/>
            <activity name="eating" length="1" intensity="2"/>
            <activity name="mission" length="9" intensity="3"/>
            <activity name="health" length="1" intensity="2"/>
            <activity name="maintenance" length="1" intensity="2"/>
            <activity name="leisure" length="2" intensity="2"/>
          </schedule>
        </crewPerson>                               
         .
         .
         .
      </crew>
\end{verbatim}

This file implements the producer/consumer relationship at the heart
of the simulation.  Each module (only the crew module is shown above)
has a name and a list of resources that it consumes and produces along with max
and desired flow rates and what source(s) produce(s) that resource
(the {\tt inputs}) or from where the resource is drawn ({\tt outputs})
More than one source or sink is allowed, but the module tries to draw as much as it can from the first.
In the above snippet some flow rates are 0 because the user has no control over
them (e.g., crew air consumption is based on crew physiology and can't
be directly controlled).  As shown above, you can have as many (or as
few) crew members as you want of various ages, weights and sexes.  You
can give a default schedule for their day (which can be changed via a
CORBA API call during simulation runs). 

The rest of the XML file specification proceeds in a similar way.  The
best way to change the configuration is to start with the default file
and edit it to reflect your requirements.  The default configuration
is based on BIO-Plex\index{BIO-Plex} \cite{tri99}, a NASA JSC ground
testbed for advanced life support.  As such, if all you want is a
``typical'' advanced life support configuration, then just use the
default XML file.

\subsection{Initializing BioSim}

Talk about alternative initialization\index{initialization} via an
API.  Again, capacities, levels, logging, etc. can be set from here.

\section{Enabling stochastic processes}
\index{stochastic}
BioSim provides the ability to change the stochastic nature of the
underlying modules via API calls or the XML initialization file.  If you want a purely deterministic
simulation you can do that.  If you want a very unpredictable
simulation you can do that as well.  This section provides an overview
of the stochastic controls available in BioSim.
The type {\tt StochasticIntensity} includes the following values:

\begin{itemize}
\item {\tt HIGH\_STOCH}
\item {\tt MEDIUM\_STOCH}
\item {\tt LOW\_STOCH}
\item {\tt NONE\_STOCH}
\end{itemize}

Each of these controls the width of the Gaussian\index{Gaussian}
function that provides the variations in output.  Setting the
stochastic intensity to NONE\_STOCH means that the simulation is
deterministic.  All noise is Gaussian.  The outputs of a module as
determined by the module's underlying equations are run through a
Gaussian filter before being output by the simulation.  The Gaussian
filter's parameters are controlled by the stochastic intensity, namely 
the higher the stochastic intensity, the higher the deviation.  

\subsection{Setting logging via the XML configuration file}
To change the stochastic intensity of a module in XML, simply add
setStochasticIntensity as an attribute to the module (if one isn't listed, it defaults to NONE\_STOCH).
example: <AirRS name="AirRS" setStochasticIntensity="HIGH\_STOCH">

The stochastic level for the entire simulation (which is equivalent to
setting each module's stochastic intensity) can by adding setStochasticIntensity to the Globals section at the head
of the XML document.
example:
<Globals crewsToWatch="CrewGroup" setStochasticIntensity="LOW\_STOCH">

\subsection{Setting logging via the API}
Each module of BioSim (i.e., the {\tt SimBioModule} class) has the
following methods:

\begin{itemize}
\item {\tt getStochasticIntensity()}: returns type {\tt
    StochasticIntensity}, which is the current stochastic level of the module.
\item {\tt setStochasticIntensity(StochasticIntensity pIntensity)}:
  Sets the current stochastic intensity of the module to {\tt
  pIntensity}.
\end{itemize}
The stochastic level for the entire simulation (which is equivalent to
setting each module's stochastic intensity) can be set via the {\tt
BioDriver} class using the {\tt setStochasticIntensity} method with
the same parameter ({\tt (StochasticIntensity})

\section{Logging}
\index{logging}

\subsection{Setting logging via the XML configuration file}

To change the logging of a module in XML, add
isLoggingEnabled as an attribute to the module (if one isn't listed, it defaults to false).
example: <AirRS name="AirRS" isLoggingEnabled="true">
You can also globally enable logging for the senors, actuators, or everything by adding
isFullLoggingEnabled, isSensorLoggingEnabled or isActuatorLoggingEnabled to the Globals section at the head
of the XML document.
example:
<Globals crewsToWatch="CrewGroup" isFullLoggingEnabled="true">

\subsection{Setting logging via the API}

Each module of BioSim (i.e., the {\tt SimBioModule} class) has the
following methods:

\begin{itemize}
\item {\tt isLogging()}: returns type {\tt
    boolean}, whether the module is logging.
\item {\tt setLogging(boolean pLogging)}:
  Sets whether the module should log.
\end{itemize}

\begin{figure}
\insertfigure{controller}{4.0in}
\caption{Interaction with the simulation.}
\label{fig:controller}
\end{figure}

\chapter{Controlling the simulation}
The previous two chapters showed you how to run BioSim, monitor its
internal state via a graphical user interface and configure it to your
specifications.  While interesting, this is not useful for control
system research.  This chapter will show you how to connect an
external control program to BioSim and change its underlying
operation.  Note that this chapter will not present an exhaustive list
of all of the objects and methods in BioSim, but only the most
important for new users.  Complete documentation of all objects and
method calls is available at:
http://www.traclabs.com/biosim/doc/api/index.html

Figure~\ref{fig:controller} shows the basic control strategy for the
simulation.  The simulation models represent the underlying equations
that govern the producer/consumer relationships (see
Section~\ref{sec:pc}).  Sensors and actuators are provided to access
the underlying simulations.  The user is responsible for writing the
controller, which uses the sensors and actuators to control the
simulation.  The user can also inject malfunctions into the underlying
simulation models and resupply consumables.   

\section{Controlling simulation runs}
The root class of the simulation is the {\tt BioDriver} class, which
contains the methods used to control the simulation.  The basic
control unit is a {\bf tick}\index{tick}, which advances the
simulation exactly one hour.  After creating your own BioDriver the
following methods are available:

\begin{enumerate}
\item  {\tt void advanceOneTick()}: Moves all modules of the
  simulation ahead exactly one hour.
\item {\tt void endSimulation()}: Ends\index{end} the simulation entirely.
\item {\tt boolean isDone()}: \index{isDone} Returns true if the simulation has met
  an end condition and has stopped.
\item {\tt void reset() }:\index{reset} Returns the simulation to its initial
  state.
\end{enumerate}

Here's an example of using one of these methods after a BioDriver
class has been created (called {\tt myBioDriver}) in a Java program:

\begin{verbatim}

    myBioDriver.advanceOneTick();

\end{verbatim}

There are a number of other methods available in the BioDriver class
that are less commonly used than those listed above.  These are
available via the JavaDoc\index{JavaDoc} facility.  After you have
installed your source code you can make documentation via the {\tt
  make-docs.sh} (for Linux) or {\tt make-docs} (for Windows)
executables in the {\tt bin} directory.  The BioDriverImpl class is
under the biosim.server.framework package.

\section{Accessing sensors and actuators}

\subsection{Generic sensors}

Sensors\index{sensors} are objects that connect to the underlying
simulation.  Each module has its own set of sensor objects, but they
all inherit the same method calls.  These method calls are:

\begin{itemize}

        \item {\tt getMax:} Returns the maximum allowed value of the sensor
        \item {\tt getMin:} Returns the minimum allowed value of the sensor
        \item {\tt getValue}: Returns the current setting of the sensor
\end{itemize}


\subsubsection{Additional sensor methods}

Several other generic methods related to sensor noise, malfunctions
and logging are available:

In addition, there will be some sensor-specific methods, which we will discuss with the appropriate sensor.

\subsection{Generic actuators}

Actuators\index{actuators} are objects that connect to the underlying simulation.  Each
module has its own set of actuator objects, but they all inherit the
same method calls.  These method calls are:

\begin{itemize}

        \item {\tt getMax}: Returns the maximum allowed value of the actuator
        \item {\tt getMin}: Returns the minimum allowed value of the actuator
        \item {\tt getValue}: Returns the current value of the actuator       
        \item {\tt setValue}: Sets the actuator

\end{itemize}

\subsubsection{Additional actuator methods}

Several other generic methods related to actuator noise, malfunctions
and logging are available:

In addition, there will be some actuator-specific methods, which we
will discuss with the appropriate actuator.


\subsection{Environments}

Environments\index{environment} contain the air\index{air} that crew
and crops use.  As there can be multiple environments in the simulation,
there can be multiple sensors tied to each environment.

\subsubsection{Sensors}
\label{sec:env_sen}
The following sensors are available in the environment:

\begin{enumerate}

\item {\tt AirInFlowRateSensor}: Flow rate in moles/tick of air into the environment
\item {\tt AirOutFlowRateSensor}: Flow rate in moles/tick of air out of the environment
\item {\tt CO2AirConcentrationSensor}: Percentage of CO2 in the environment.
\item {\tt CO2AirStoreInFlowRateSensor}: Flow rate in liters/tick of CO2 from a store. (paired with CO2AirEnvironmentInFlowRateSensor)
\item {\tt CO2AirEnvironmentInFlowRateSensor}: Flow rate in liters/tick of CO2 into the environment (paired with CO2AirStoreInFlowRateSensor)
\item {\tt CO2AirPressureSensor}: Pressure in kPA of CO2 in the environment
\item {\tt CO2AirStoreOutFlowRateSensor}: Flow rate in liters/tick of CO2 from a store. (paired with CO2AirEnvironmentOutFlowRateSensor)
\item {\tt CO2AirEnvironmentOutFlowRateSensor}: Flow rate in liters/tick of CO2 out of the environment (paired with CO2AirStoreOutFlowRateSensor)
\item {\tt NitrogenAirConcentrationSensor}: Percentage of Nitrogen in the environment
\item {\tt NitrogenAirEnvironmentInFlowRateSensor}: Flow rate in liters/tick of nitrogen into the environment (paired with NitrogenAirStoreInFlowRateSensor)
\item {\tt NitrogenAirStoreInFlowRateSensor}: Flow rate in liters/tick of nitrogen into a store (paired with NitrogenAirEnvironmentInFlowRateSensor)
\item {\tt NitrogenAirEnvironmentOutFlowRateSensor}: Flow rate in liters/tick of nitrogen out of an environment (paired with NitrogenAirStoreOutFlowRateSensor)
\item {\tt NitrogenAirStoreOutFlowRateSensor}: Flow rate in liters/tick of nitrogen out of a store (paired with NitrogenAirEnvironmentOutFlowRateSensor)
\item {\tt NitrogenAirPressureSensor}: Pressure in kPA of nitrogen in the environment
\item {\tt O2AirConcentrationSensor}: Percentage of O2 in the environment.
\item {\tt O2AirEnvironmentInFlowRateSensor}: Flow rate in liters/tick of O2 into the environment (paired with O2AirStoreInFlowRateSensor)
\item {\tt O2AirStoreInFlowRateSensor}: Flow rate in liters/tick of O2 out of the store (paired with O2AirEnvironmentInFlowRateSensor)
\item {\tt O2AirEnvironmentOutFlowRateSensor}: Flow rate in liters/tick of O2 out of the environment (paired with O2AirStoreOutFlowRateSensor)
\item {\tt O2AirStoreOutFlowRateSensor}: Flow rate in liters/tick of O2 out of the store (paired with O2AirEnvironmentOutFlowRateSensor)
\item {\tt O2AirPressureSensor}: Pressure in kPA of O2 in the environment
\item {\tt OtherAirConcentrationSensor}: Percentage of trace gases in the environment.
\item {\tt OtherAirPressureSensor}: Pressure in kPA of trace gases in the environment
\item {\tt WaterAirConcentrationSensor}: Percentage of water (humidity) in the environment
\item {\tt WaterAirEnvironmentInFlowRateSensor}: Flow rate in liters/tick of water vapor into the environment (paired with WaterAirStoreInFlowRateSensor)
\item {\tt WaterAirStoreInFlowRateSensor}: Flow rate in liters/tick of water vapor into the store (paired with WaterAirEnvironmentInFlowRateSensor)
\item {\tt WaterAirEnvironmentOutFlowRateSensor}: Flow rate in liters/tick of water vapor out of the environment (paired with WaterAirStoreOutFlowRateSensor)
\item {\tt WaterAirStoreOutFlowRateSensor}: Flow rate in liters/tick of water vapor out of the store (paired with WaterAirEnvironmentOutFlowRateSensor)
\item {\tt WaterAirPressureSensor}: Pressure in kPA of water in the environment
\end{enumerate}

\subsubsection{Actuators}

\begin{enumerate}

\item {\tt AirInFlowRateActuator}: Flow rate in moles/tick of air into the environment
\item {\tt AirOutFlowRateActuator}: Flow rate in moles/tick of air out of the environment
\item {\tt CO2AirStoreInFlowRateActuator}: Flow rate in liters/tick of CO2 from a store. (paired with CO2AirEnvironmentInFlowRateActuator)
\item {\tt CO2AirEnvironmentInFlowRateActuator}: Flow rate in liters/tick of CO2 into the environment (paired with CO2AirStoreInFlowRateActuator)
\item {\tt CO2AirStoreOutFlowRateActuator}: Flow rate in liters/tick of CO2 from a store. (paired with CO2AirEnvironmentOutFlowRateActuator)
\item {\tt CO2AirEnvironmentOutFlowRateActuator}: Flow rate in liters/tick of CO2 out of the environment (paired with CO2AirStoreOutFlowRateActuator)
\item {\tt NitrogenAirEnvironmentInFlowRateActuator}: Flow rate in liters/tick of nitrogen into the environment (paired with NitrogenAirStoreInFlowRateActuator)
\item {\tt NitrogenAirStoreInFlowRateActuator}: Flow rate in liters/tick of nitrogen into a store (paired with NitrogenAirEnvironmentInFlowRateActuator)
\item {\tt NitrogenAirEnvironmentOutFlowRateActuator}: Flow rate in liters/tick of nitrogen out of an environment (paired with NitrogenAirStoreOutFlowRateActuator)
\item {\tt NitrogenAirStoreOutFlowRateActuator}: Flow rate in liters/tick of nitrogen out of a store (paired with NitrogenAirEnvironmentOutFlowRateActuator)
\item {\tt O2AirEnvironmentInFlowRateActuator}: Flow rate in liters/tick of O2 into the environment (paired with O2AirStoreInFlowRateActuator)
\item {\tt O2AirStoreInFlowRateActuator}: Flow rate in liters/tick of O2 out of the store (paired with O2AirEnvironmentInFlowRateActuator)
\item {\tt O2AirEnvironmentOutFlowRateActuator}: Flow rate in liters/tick of O2 out of the environment (paired with O2AirStoreOutFlowRateActuator)
\item {\tt O2AirStoreOutFlowRateActuator}: Flow rate in liters/tick of O2 out of the store (paired with O2AirEnvironmentOutFlowRateActuator)
\item {\tt WaterAirEnvironmentInFlowRateActuator}: Flow rate in liters/tick of water vapor into the environment (paired with WaterAirStoreInFlowRateActuator)
\item {\tt WaterAirStoreInFlowRateActuator}: Flow rate in liters/tick of water vapor into the store (paired with WaterAirEnvironmentInFlowRateActuator)
\item {\tt WaterAirEnvironmentOutFlowRateActuator}: Flow rate in liters/tick of water vapor out of the environment (paired with WaterAirStoreOutFlowRateActuator)
\item {\tt WaterAirStoreOutFlowRateActuator}: Flow rate in liters/tick of water vapor out of the store (paired with WaterAirEnvironmentOutFlowRateActuator)

\end{enumerate}

\subsubsection{Malfunctions}
State exactly what the malfunctions do.  How do the method calls work.
What is the malfunction. What is the leak rate, etc. for each of the
different malfunction classes.  Since method calls are probably
identical may want to move this up to top and then use this section
only for module-specific malfunction information.

\subsection{Crew}
\index{crew}
\subsubsection{Sensors}

\begin{enumerate}

\item {\tt CrewGroupActivitySensor}:  Returns average activity level of crew
\item {\tt CrewGroupAnyDeadSensor}: Returns whether any crew member has died
\item {\tt CrewGroupDeathSensor}: Return whether the entire crew has died
\item {\tt CrewGroupProductivitySensor}: Total productivity of the crew

\end{enumerate}

\subsubsection{Actuators}

\begin{enumerate}

\item {\tt CrewGroupActivityActuator}:

\end{enumerate}

\subsubsection{Malfunctions}

\subsection{Air}
\index{air}
\subsubsection{Sensors}

\begin{enumerate}
\item {\tt CO2InFlowRateSensor}: Reports the CO2 flow rate (in moles) into a module
\item {\tt CO2OutFlowRateSensor}: Reports the CO2 flow rate (in moles) out of a module
\item {\tt CO2StoreLevelSensor}: How much CO2 (in moles) is contained in this store.
\item {\tt H2InFlowRateSensor}: Reports the H2 flow rate (in moles) into a module
\item {\tt H2OutFlowRateSensor}: Reports the H2 flow rate (in moles) out of a module
\item {\tt H2StoreLevelSensor}: How much   (in moles) is contained in this store.
\item {\tt NitrogenInFlowRateSensor}: Reports the nitrogen flow rate (in moles) into a module
\item {\tt NitrogenOutFlowRateSensor}: Reports the nitrogen flow rate (in moles) out of a module
\item {\tt NitrogenStoreLevelSensor}: How much nitrogen (in moles) is contained in this store.
\item {\tt O2InFlowRateSensor}: Reports the O2 flow rate (in moles) into a module
\item {\tt O2OutFlowRateSensor}: Reports the O2 flow rate (in moles) out of a module
\item {\tt O2StoreLevelSensor}: How much O2 (in moles) is contained in this store.
\end{enumerate}

\subsubsection{Actuators}

\begin{enumerate}
\item {\tt CO2InFlowRateAcutator}: Changes how much CO2 (in moles) goes into the module
\item {\tt CO2OutFlowRateAcutator}: Changes how much CO2 (in moles) goes out of the module
\item {\tt H2InFlowRateAcutator}: Changes how much H2 (in moles) goes into the module
\item {\tt H2OutFlowRateAcutator}: Changes how much H2 (in moles) goes out of the module
\item {\tt NitrogenInFlowRateAcutator}: Changes how much nitrogen (in moles) goes into the module
\item {\tt NitrogenOutFlowRateAcutator}: Changes how much nitrogen (in moles) goes out of the module
\item {\tt O2InFlowRateAcutator}: Changes how much O2 (in moles) goes into the module
\item {\tt O2OutFlowRateAcutator}: Changes how much O2 (in moles) goes out of the module
\end{enumerate}

\subsubsection{Malfunctions}

Same as others


\subsection{Water}
\index{water}

\subsubsection{Sensors}
\begin{enumerate}
\item {\tt DirtyWaterInFlowRateSensor}: Reports the dirty water flow rate (in liters) into a module
\item {\tt DirtyWaterOutFlowRateSensor}: Reports the dirty water flow rate (in liters) out of a module
\item {\tt DirtyWaterStoreLevelSensor}: How much dirty water (in liters) is contained in this store.
\item {\tt GreyWaterInFlowRateSensor}: Reports the grey water flow rate (in liters) into a module
\item {\tt GreyWaterOutFlowRateSensor}: Reports the grey water flow rate (in liters) out of a module
\item {\tt GreyWaterStoreLevelSensor}: How much grey water (in liters) is contained in this store.
\item {\tt PotableWaterInFlowRateSensor}: Reports the potable water flow rate (in liters) into a module
\item {\tt PotableWaterOutFlowRateSensor}: Reports the potable water flow rate (in liters) out of a module
\item {\tt PotableWaterStoreLevelSensor}: How much potable water (in liters) is contained in this store.
\end{enumerate}

\subsubsection{Actuators}

\begin{enumerate}
\item {\tt DirtyWaterInFlowRateActuator}: Changes how much dirty water (in liters) goes into the module
\item {\tt DirtyWaterOutFlowRateActuator}: Changes how much dirty water (in liters) goes ouf of the module
\item {\tt GreyWaterInFlowRateActuator}: Changes how much grey water (in liters) goes into the module
\item {\tt GreyWaterOutFlowRateActuator}: Changes how much grey water (in liters) goes ouf of the module
\item {\tt PotableWaterInFlowRateActuator}: Changes how much potable water (in liters) goes into the module
\item {\tt PotableWaterOutFlowRateActuator}: Changes how much potable water (in liters) goes ouf of the module
\end{enumerate}

\subsubsection{Malfunctions}

\subsection{Biomass}
\index{biomass}

\subsubsection{Sensors}
\begin{enumerate}
\item {\tt BiomassInFlowRateSensor}: Reports the biomass flow rate (in kilograms) into a module
\item {\tt BiomassOutFlowRateSensor}: Reports the biomass flow rate (in kilograms) out of a module
\item {\tt BiomassStoreLevelSensor}: How much biomass (in kilograms) is contained in this store.
\item {\tt FoodInFlowRateSensor}: Reports the food flow rate (in kilograms) into a module
\item {\tt FoodOutFlowRateSensor}: Reports the food flow rate (in kilograms) out of a module
\item {\tt FoodStoreLevelSensor}: How much food (in kilograms) is contained in this store.
\end{enumerate}

\subsubsection{Actuators}
\begin{enumerate}
\item {\tt BiomassInFlowRateActuator}: Changes how much biomass (in kilograms) goes into the module
\item {\tt BiomassOutFlowRateActuator}: Changes how much biomass (in kilograms) goes out of the module
\item {\tt FoodInFlowRateActuator}: Changes how much food (in kilograms) goes into the module
\item {\tt FoodOutFlowRateActuator}: Changes how much food (in kilograms) goes out of the module
\end{enumerate}

\subsubsection{Malfunctions}

\subsection{Dry Waste}
\index{waste}
\subsubsection{Sensors}
\begin{enumerate}
\item {\tt DryWasteInFlowRateSensor}: Changes how much dry Waste (in kilograms) goes into the module
\item {\tt DryWasteOutFlowRateSensor}: Changes how much dry Waste (in kilograms) goes out of the module
\item {\tt DryWasteStoreLevelSensor}: How much dry Waste (in kilograms) is contained in this store.
\end{enumerate}

\subsubsection{Actuators}
\begin{enumerate}
\item {\tt DryWasteInFlowRateActuator}: Changes how much dry waste (in kilograms) goes into the module
\item {\tt DryWasteOutFlowRateActuator}: Changes how much dry waste (in kilograms) goes out of the the module
\end{enumerate}

\subsubsection{Malfunctions}

\subsection{Power}
\index{power}

\subsubsection{Sensors}

\begin{enumerate}
\item {\tt PowerInFlowRateSensor}: Changes how much power (in watts) goes into the module
\item {\tt PowerOutFlowRateSensor}: Changes how much power (in watts) goes out of the module
\item {\tt PowerStoreLevelSensor}: How much power (in watts) is contained in this store.
\end{enumerate}

\subsubsection{Actuators}

\begin{enumerate}
\item {\tt PowerInFlowRateActuator}: Changes how much power (in watts) goes into the module
\item {\tt PowerOutFlowRateActuator}: Changes how much power (in watts) goes out of the module
\end{enumerate}

\subsubsection{Malfunctions}



\chapter{Writing a simple Java controller}

A short tutorial on creating a BioSim, initializing it and then
calling some sensors, actuators and ticks.  Assume JacOrb. 
See BIOSIM/doc/TestBiosim.java

\chapter{Writing a simple C++ controller}



Same as above in C++.  Assume ACEOrb.  Assume Linux (?).

\chapter{Control Examples}

Include GA, Rice learners, Texas Tech learner.

\chapter{Conclusions}

\appendix 
\chapter{Java Example}

\begin{verbatim}

Put parts of Rice handcontroller here

\end{verbatim}

\chapter{C++ Example}

\begin{verbatim}

#include "biosimC.h"
#include <CosNamingC.h>

static const char* CorbaServerName = "BioDriver0";
int
main(int argc,char *argv[])
{
  try {
    // Initialize orb
    CORBA::ORB_var orb = CORBA::ORB_init(argc,argv,"TestClient");

    cerr << "initialized orb" << endl;

    if (argc < 2) {
      cerr << "Usage: " << argv[0]
           << " IOR_string" << endl;
      return 1;
    }

// Use the first argument to create the object reference,
    // in real applications we use the naming service, but let's do
    // the easy part first!

    CORBA::Object_var obj =
      orb->string_to_object (argv[1]); 

    cout << "After resolve_initial_references" << endl;


    /*   CORBA::Object_var obj =
         orb->string_to_object (argv[1]); */
 
    CosNaming::NamingContext_var inc = CosNaming::NamingContext::_narrow(obj);

    cout << "After NameService narrow" << endl;

    if(CORBA::is_nil(inc)) {
      cerr << "Unable to narrow NameService." << endl;
      throw 0;
    }  // end if couldn't get root naming context
    
    CosNaming::Name name;
    name.length(1);
    name[0].id = CORBA::string_dup(CorbaServerName);
    name[0].kind = CORBA::string_dup("");

    CORBA::Object_var tmp = inc->resolve(name);

    cout << "After resolve name" << endl;

    if(CORBA::is_nil(tmp)) {
      cerr << "Unable to resolve " << CorbaServerName << " from NameService" << endl;
      throw 0;
    }  // end if couldn't find object

    cerr << "got object" << endl;

    // Now downcast the object reference to the appropriate type

    biosim::idl::framework::BioDriver_var biodriver = biosim::idl::framework::BioDriver::_narrow(tmp);


    cerr << "narrowed" << endl;

    // code here

    biodriver->spawnSimulationTillDead();

    cerr << "spawned" << endl;

    orb->destroy ();
  }
  catch (CORBA::Exception &ex) {
    std::cerr << "CORBA exception raised! " << std::endl;
  }
  return 0;
}

\end{verbatim}

\bibliography{biosim}
\bibliographystyle{plain}

\begin{theindex}

\input{users_manual.ind}

\end{theindex}


\end{document}



\end{theindex}


\end{document}



\end{theindex}


\end{document}



\end{theindex}

\end{document}

